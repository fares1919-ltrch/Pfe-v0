\chapter{Release 1 : Gérer utilisateurs}
\label{chap_sprint1}
\addcontentsline{toc}{section}{Introduction}
\section*{Introduction}
\label{sec_introduction}
Dans cette partie, nous traitons les points suivants : tout d'abord, l'organisation et le sprint backlog. Ensuite, nous décrivons le contenu du premier sprint, intitulé "Gérer utilisateur". Nous entamons ensuite la phase d'analyse et nous explorons les solutions conceptuelles. Enfin, nous présentons les différentes réalisations obtenues.

\subsection{Organisation}
Le tableau 3.1 ci-dessous donne un aperçu détaillé sur le Backlog du premier sprint qui prend en
charge les fonctionnalités « s'inscrire », « s'authentifier », « récupérer mot de passe » , « Gérer profil » et « gérer officers et managers ».

\begin{longtable}
{|>{\centering\arraybackslash}p{0.7cm}
 |>{\arraybackslash}p{5cm}
 |>{\centering\arraybackslash}p{1.2cm}
 |>{\arraybackslash}p{7cm}
 |>{\centering\arraybackslash}p{1.5cm}|}

\caption{\centering Backlog du sprint 1 : « Gérer utilisateurs »}
\label{tab:backlog:ch2:1} \\

\hline
\rowcolor{gray!30}
ID & User Story & ID & Tâche & Durée/j \\
\hline
\endfirsthead
\endhead

\hline
\endfoot

\hline
\endlastfoot

\multirow{3}{0.7cm}{1.1}
& \multirow{3}{5cm}{En tant qu'internaute, je souhaite pouvoir m'inscrire.}
& 1.1.1 & Créer les interfaces d'inscription. & 1 \\
\cline{3-5}
& & 1.1.2 & Développer une API pour vérifier la cohérence des données fournies. & 1 \\
& & 2.1.3 & Tester la fonctionnalité d'inscription. & 1 \\
\hline

\multirow{3}{0.7cm}{2.2}
& \multirow{3}{5cm}{En tant que manager, citoyen ou officier, je dois m'authentifier.}
& 2.2.1 & Développer les interfaces d'authentification. & 1 \\
\cline{3-5}
& & 2.2.2 & Créer une API pour valider les données d'authentification. & 1 \\
\cline{3-5}
& & 2.2.3 & Tester la fonctionnalité d'authentification. & 1 \\
\hline

\multirow{4}{0.7cm}{3.1}
& \multirow{4}{5cm}{En tant que manager, citoyen ou officier, je peux récupérer mon mot de passe.}
& 3.1.1 & Créer les interfaces de récupération du mot de passe. & 1 \\
\cline{3-5}
& & 3.1.2 & Développer une API pour valider les données lors de la récupération et envoyer un email de réinitialisation. & 1 \\
\cline{3-5}
& & 3.1.3 & Mettre en place l'envoi automatique d'un e-mail de récupération. & 1 \\
\cline{3-5}
& & 3.1.4 & Tester la fonctionnalité de récupération de mot de passe. & 1 \\
\hline

\multirow{4}{0.7cm}{4.1}
& \multirow{4}{5cm}{En tant que manager, citoyen ou officier, je peux consulter, modifier mes informations, révoquer les sessions actives et supprimer mon compte.}
& 4.1.1 & Créer l'interface de gestion de profil. & 1 \\
\cline{3-5}
& & 4.1.2 & Développer une API de gestion de profil. & 1 \\
\cline{3-5}
& & 4.1.3 & Tester la fonctionnalité de gestion de profil. & 1 \\
\hline

\multirow{3}{0.7cm}{5.1}
& \multirow{3}{5cm}{En tant qu'administrateur, je souhaite gérer les managers et officiers,je peux consulter, activer,bloquer,débloquer ou supprimer les comptes des managers et officiers.}
& 5.1.1 & Développer l'interface de gestion des managers et officiers. & 2 \\
\cline{3-5}
& & 5.1.2 & Développer une API pour récupérer la liste des managers et officiers et modifier le statut de leurs comptes. & 2 \\
\cline{3-5}
& & 5.1.3 & Tester la fonctionnalité de gestion des managers et des officiers. & 1 \\
\hline

\end{longtable}


\subsection{Analyse}

Durant cette phase d'analyse, nous approfondissons les diverses fonctionnalités en les accompagnant de leurs cas d'utilisation respectifs.
\subsubsection{Diagrammes de cas d'utilisation}
Nous présentons dans cette partie les différents cas d'utilisation raffinés.\\

\textbf{ – Raffinement de cas d'utilisation « S'inscrire »}\\
La \hyperref[fig:3.1]{figure 3.1} montre le raffinement de cas d'utilisation « S'inscrire ».  qui permet à un internaute de s'inscrire afin de tirer profit des fonctionnalités offertes par \textbf{Identity Secure}.
\begin{figure}[H]
\centering
\includegraphics [width=\textwidth]{chapitre2/inscrire/UC Signup}
\caption{ Sprint 1 -Diagramme de cas d'utilisation « s'inscrire »}
\label{fig:3.1}
\end{figure}

Le Tableau 3.2 représente une description textuelle du cas d'utilisation « S'inscrire ». Il détaille le scénario nominal ainsi que les enchaînements alternatifs.
\begin{longtable}{|>{\arraybackslash}p{4.2cm}|>{\arraybackslash}p{12.5cm}|}
\caption{\centering Description textuelle du sous cas d'utilisation «S'inscrire»}
\label{tab:backlog:ch2:inscription} \\
\hline
\rowcolor{gray!30}
\textbf{Cas d'utilisation} & S'inscrire \\
\hline
\textbf{Acteur} & Internaute \\
\hline
\textbf{Résumé} &
\begin{itemize}[label=]
    \item \textbf{Scénario 1: S'inscrire via le formulaire}
    \item L'internaute s'inscrit via le formulaire d'inscription et valide son adresse email pour tirer profit des fonctionnalités de

\end{itemize} \\
\hline
\textbf{} &
\begin{itemize}[label=]
    \item «Identity-Secure».
    \item \textbf{Scénario 2: S'inscrire avec Google}
    \item L'internaute s'inscrit en utilisant son compte Google pour accéder rapidement aux fonctionnalités de «Identity-Secure».
\end{itemize}\\
\hline
\textbf{Pré-conditions} &
\begin{itemize}[label=]
    \item \textbf{Scénario 1: S'inscrire via le formulaire}
    \item 1- L'internaute accède à la page « accueil ».
    \item 2- L'internaute clique sur le bouton « S'inscrire ».
    \item 3- Le système affiche le formulaire «Create Account».
    \item \textbf{Scénario 2: S'inscrire avec Google}
    \item 1- L'internaute accède à la page « accueil ».
    \item 2- L'internaute clique sur le bouton « S'inscrire avec google ».
    \item 3- Le système affiche le formulaire «Create Account».
\end{itemize} \\
\hline
\textbf{Description de scénario nominal} &
\begin{itemize}[label=]
    \item \textbf{Scénario 1: S'inscrire via le formulaire}
    \item 1- L'internaute remplit le formulaire avec : nom d'utilisateur, adresse e-mail, mot de passe.
    \item 2- Il accepte les conditions d'utilisation et la politique de confidentialité.
    \item 3- Il clique sur « Create Account ».
    \item 4- Le système vérifie les données.
    \item 5- Vérification d'unicité du nom d'utilisateur.
    \item 6- Vérification d'unicité de l'adresse e-mail.
    \item 7- Enregistrement des données avec rôle "utilisateur" et statut "non vérifié".


\end{itemize} \\
\hline
\textbf{}&
\begin{itemize}[label=]
    \item 8- Génération d'un token de vérification et envoi d'un email.
    \item 9- Message informant l'envoi de l'email.
    \item 10- L'internaute ouvre l'email de validation.
    \item 11- Il clique sur "Valider mon compte".
    \item 12- Vérification du token.
    \item 13- Mise à jour du compte en "vérifié".
    \item 14- L'internaute peut désormais se connecter.
    \item \textbf{Scénario 2: S'inscrire avec Google}
    \item 1- L'internaute clique sur « Sign Up With Google ».
    \item 2- Redirection vers la page d'authentification Google.
    \item 3- Choix ou saisie des identifiants Google.
    \item 4- Vérification de l'existence de l'email dans la base.
    \item 5- Authentification Google et réception des infos.
    \item 6- Création d'un compte avec rôle "Citoyen" et statut "vérifié".
    \item 7- Génération d'un token d'accès.
    \item 8- Redirection vers le tableau de bord selon rôle.
\end{itemize}\\
\hline
\textbf{Enchaînements alternatifs} &
\begin{itemize}[label=]
    \item \textbf{Scénario 1: S'inscrire via le formulaire}
    \item 4.1- Données invalides : message « Please fill in all required fields correctly. » — retour au point 1 du scénario nominal.
    \item 4.2- Mots de passe non correspondants : message « Passwords do not match. » — retour au point 1.
    \item 4.3- Mot de passe trop faible : message « Password must be at least 8 characters long and contain at least one uppercase letter, one lowercase letter, one number, and one special character. » —> retour au point 1.

\end{itemize} \\
\hline
\textbf{}&
\begin{itemize}[label=]
    \item 5.1- Nom d'utilisateur déjà pris : message « Username is already taken. » —> retour au point 1.
    \item 6.1- E-mail déjà utilisé : message « This email is already in use. » — retour au point 1.
    \item 10.1- E-mail de validation non reçu : possibilité de renvoi depuis la page de connexion.
    \item 12.1- Token expiré après 48h : message d'erreur et proposition d'un nouveau lien.
    \item 15.1- Tentative de connexion sans validation email : message « Please verify your email address before logging in. » avec option de renvoi.
    \item \textbf{Scénario 2: S'inscrire avec Google}
    \item 5.1- E-mail Google déjà associé : connexion au compte existant.
    \item 6.1- Erreur lors de la création du compte : message « Failed to complete authentication. Please try again. » et redirection vers la page d'inscription.
\end{itemize}\\
\hline
\textbf{Post-conditions} & Un nouveau compte utilisateur est créé, vérifié par e-mail, et l'internaute peut se connecter au système. \\
\hline
\end{longtable}






\clearpage
\textbf{ – Raffinement de cas d'utilisation « S'authentifier »}
\\
La \hyperref[fig:3.2]{figure 3.2} montre le raffinement de cas d'utilisation « S'authentifier». L'acteur doit
s'authentifier afin de tirer profit des fonctionnalités offertes par « Identity-Secure ».
\begin{figure}[H]
\centering
\includegraphics [width=\textwidth]{chapitre2/auth/UC signin-use.png}
\caption{ Sprint 1 -Diagramme de cas d'utilisation « s'authentifier » }
\label{fig:3.2}
\end{figure}

Le Tableau 3.3 représente une description textuelle du cas d'utilisation « S'authentifier ». Il détaille le scénario nominal ainsi que les enchaînements alternatifs.


\begin{longtable}{|>{\arraybackslash}p{4.2cm}|>{\arraybackslash}p{12.5cm}|}
\caption{\centering Description textuelle du sous-cas d'utilisation « S'authentifier »}
\label{tab:backlog:ch2:3} \\
\hline
\rowcolor{gray!30}
\textbf{Cas d'utilisation} & S'authentifier \\
\hline
\endfirsthead

\hline
\endhead

\hline
\endfoot

\hline \hline
\endlastfoot

\textbf{Acteurs} & Citoyen, Manager, Officer \\
\hline

\textbf{Résumé} &
\begin{itemize}[label=]
  \item \textbf{Scénario 1 :} S'authentifier avec formulaire.
  \item L'acteur se connecte à la plateforme Identity-Secure via un formulaire classique.
  \item \textbf{Scénario 2 :} S'authentifier avec Google.
  \item L'acteur se connecte à la plateforme en utilisant son compte Google.
\end{itemize} \\
\hline

\textbf{Pré-conditions} &
\begin{itemize}[label=]
  \item \textbf{Scénario 1 :}
  \item 1.1 L'acteur accède à la page « accueil ».
    \item 1.2 L'acteur a déjà créé un compte.
    \item 1.3 L'acteur a vérifié son compte.
    \item 1.4 L'acteur clique sur le bouton « Sign In ».
    \item 1.5 Le système affiche le formulaire d'authentification.

  \item \textbf{Scénario 2 :}
    \item 2.1 L'acteur accède à la page « accueil ».
    \item 2.2 L'acteur a déjà un compte ou possède un compte Google.
    \item 2.3 L'acteur clique sur le bouton « Sign In ».
    \item 2.4 Le système affiche le formulaire d'authentification.
\end{itemize} \\
\hline

\textbf{Description de scénario nominal} &
\begin{itemize}[label=]
  \item \textbf{Scénario 1 :}
    \item 1.1 L'acteur saisit son nom d'utilisateur.
    \item 1.2 L'acteur saisit son mot de passe.
    \item 1.3 L'acteur peut cocher l'option « Remember me ».
    \item 1.4 L'acteur clique sur le bouton « Sign In ».
    \item 1.5 Le système vérifie l'existence du compte.
    \item 1.6 Le système vérifie les coordonnées.
    \item 1.7 Le système vérifie la validation de l'email.
    \item 1.8 Le système génère un jeton d'accès (token).
    \item 1.9 Le système redirige l'acteur selon son rôle
  \item \textbf{Scénario 2 :}
        \item 2.1 L'acteur clique sur le bouton «Sign In With Google».
\item 2.2 Le système redirige vers la page d'authentification Google.
\end{itemize} \\
\hline



\textbf{} &
\begin{itemize}

    \item 2.3 L'acteur sélectionne ou saisit son compte Google.
    \item 2.4 Google authentifie l'acteur et transmet les infos au système.
    \item 2.5 Le système vérifie l'association de l'email Google.
    \item 2.6 Le système génère un jeton d'accès (token).
    \item 2.7 Le système redirige l'acteur selon son rôle.
\end{itemize} \\
\hline





\textbf{Enchaînements alternatifs} &
\begin{itemize}[label=]
  \item \textbf{Scénario 1 :}
    \item 1.1 Champ nom d'utilisateur vide → message : « Username is required » → retour à 1.1
    \item 1.2 Champ mot de passe vide → le système affiche un message : « Password is required » → retour à 1.2
    \item 1.5 Nom d'utilisateur inexistant → le système affiche le message : « Login failed. Please check your credentials. » → retour à 1.1
    \item 1.6 Mot de passe incorrect → même message → retour à 1.2
    \item 1.7 E-mail non vérifié → message : « Please verify your email address before logging in. » avec options :
    \begin{itemize}[label=]
      \item Bouton « Resend verification email »
      \item Lien vers la page de connexion
    \end{itemize}
    \item 1.7 L'acteur clique sur « Resend verification email » → un nouvel email est envoyé → le système affiche un message de confirmation.
\item \textbf{Scénario 2 :}
    \item 2.5 E-mail Google non associé à un compte → le système crée un compte avec le rôle « utilisateur » et redirige vers « citizen-dashboard ».
    \item 2.6 Erreur d'authentification Google → message :
\end{itemize} \\
\hline
\textbf{}{}&
\begin{itemize}[label=]

    \item « Failed to complete authentication. Please try again. » → redirection vers la page de connexion.
\end{itemize}\\
\hline
\textbf{Post-conditions} & L'acteur est authentifié et accède aux fonctionnalités correspondant à son rôle. \\
\hline
\end{longtable}


\vspace{1 cm}


\textbf{ – Raffinement de cas d'utilisation « Récupérer mot de passe »}
\\
La \hyperref[fig:3.3]{figure 3.3} montre le raffinement de cas d'utilisation « Récupérer mot de passe ». Si l'acteur oublie son mot de
passe, « Identity-Secure » fournit un service de récupération sécurisé.
\begin{figure}[H]
\centering
\includegraphics[width=\textwidth] {chapitre2/Forgot/UC Récupérer mot de passe.png}
\caption{ Sprint 1 -Diagramme de cas d'utilisation « Récupérer mot de passe »
}
\label{fig:3.3}
\end{figure}

Le Tableau 3.4 représente une description textuelle du cas d'utilisation « Récupérer Mot de Passe ». Il détaille le scénario nominal ainsi que les enchaînements alternatifs.
\begin{longtable}{|>{\arraybackslash}p{4.2cm}|>{\arraybackslash}p{12.5cm}|}
\caption{\centering Description textuelle du cas d'utilisation « Récupérer Mot de Passe »}
\label{tab:backlog} \\
\hline
\rowcolor{gray!30}
\textbf{Cas d'utilisation} &  Récupérer Mot de Passe \\
\hline
\endfirsthead

\hline
\endhead

\hline
\endfoot

\hline \hline
\endlastfoot
\textbf{Acteur} & Citoyen, Manager, Officer \\
\hline
\textbf{Résumé} &  L'acteur peut récupérer son mot de passe pour s'authentifier \\
\hline
\textbf{Pré-conditions} &

\begin{itemize}[label=]
\item{1-} L'utilisateur accède à la page « accueil ».
\item{2-} L'utilisateur a déjà créé un compte.
\item{3-} L'acteur a vérifié son compte.
\item{4-} L'acteur accède à l'interface de « Sign In ».
\item{5-} L'acteur clique sur « Forgot Password? ».
\end{itemize}

\\
\hline
\textbf{Description de scénario nominal }  &
\begin{itemize}[label=]
\item{1-} Le système affiche l'interface « Reset Password ».
\item{2-} L'acteur saisit son adresse e-mail.
\item{3-} L'acteur clique sur le bouton « Send Reset Link ».
\item{4-} Le système vérifie la validité de l'adresse e-mail saisie.
\item{5-} Le système vérifie l'existence de l'adresse e-mail dans la base de données.
\item{6-} Le système génère un token de réinitialisation et l'envoie à l'adresse e-mail de l'acteur.
\item{7-} Le système redirige l'acteur vers l'interface de vérification du code.
\item{8-} L'acteur accède à sa boîte de réception et clique sur le lien de réinitialisation ou copie le code de vérification.
\item{9-} Si l'acteur a cliqué sur le lien, le système le redirige directement vers l'interface « New Password ».
\item{10-} Si l'acteur a copié le code, il le saisit dans l'interface de vérification et clique sur « Verify Code ».
    \item{11-} Le système vérifie la validité du code et redirige l'acteur vers l'interface « New Password ».
    \item{12-} L'acteur saisit son nouveau mot de passe, le confirme et clique sur « Reset Password ».

\end{itemize}\\
\hline
\textbf{} &

\begin{itemize}[label=]

    \item{13-} Le système vérifie la validité et la cohérence du nouveau mot de passe.
    \item{14-} Le système met à jour le mot de passe de l'acteur dans la base de données.
    \item{15-} Le système affiche un message « Your password has been reset successfully. You can now login with your new password. » et redirige l'acteur vers la page de connexion.
\end{itemize} \\
\hline

\textbf{Enchaînements alternatifs} &
\begin{itemize}[label=]
  \item{4.1-} L'adresse email est vide ou non valide : le système affiche un message d'erreur et l'acteur reste sur la même page.
    \item{5.1-} L'adresse email n'existe pas dans la base de données : le système envoie quand même une confirmation pour des raisons de sécurité, mais aucun email n'est réellement envoyé.
    \item{8.1-} L'acteur ne reçoit pas le code ou le lien : l'acteur peut cliquer sur « Resend Code » après un délai de 30 secondes.
    \item{10.1-} Le code saisi est invalide : le système affiche un message d'erreur « Invalid verification code » et l'acteur reste sur la même page.
    \item{11.1-} Le token a expiré (délai de 1 heure) : le système affiche un message d'erreur « Reset token has expired. Please request a new one. »
    et redirige l'acteur vers la page de demande de réinitialisation.
    \item{13.1-} Les mots de passe ne correspondent pas : le système affiche un message d'erreur « Passwords do not match. » et l'acteur reste sur la même page.
      \item{13.2-} Le nouveau mot de passe ne respecte pas les critères de sécurité : le système affiche un message d'erreur « Please enter
 \end{itemize}


\\
\hline
\textbf{} &

\begin{itemize}[label=]
    \item  and confirm your new password. » et l'acteur reste sur la même page.
      \item{13.3-} Le token est manquant : le système affiche un message d'erreur « Reset token is missing. Please use the link from your email. » et l'acteur est redirigé vers la page de demande de réinitialisation.

\end{itemize} \\
\hline
\textbf{Post-conditions} & Le mot de passe a été bien modifié et l'acteur peut se connecter avec son nouveau mot de passe.

\end{longtable}
\textbf{ – Raffinement de cas d'utilisation « Gérer managers et officiers »}\\
La \hyperref[fig:3.4]{figure 3.4} montre le raffinement de cas d'utilisation « Gérer managers et officiers » qui permet à l'administrateur de gérer la liste des managers et officiers inscrits ,consulter la liste ,activer leurs comptes et gérer le statut de leurs comptes et supprimer leurs comptes en les bloquant ou les débloquant.

\begin{figure}[H]
\centering
\includegraphics [width=\textwidth]{chapitre2/Admin/GererRoleUC.png}
\caption{ Sprint 1 - Diagramme de cas d'utilisation « Gérer managers et officiers »}
\label{fig:3.12}
\end{figure}

Le Tableau 3.6 représente une description textuelle du cas d'utilisation « Gérer managers et officiers ». Il détaille le scénario nominal ainsi que les enchaînements alternatifs.

\begin{longtable}{|>{\arraybackslash}p{4.2cm}|>{\arraybackslash}p{12.5cm}|}
\caption{\centering Description textuelle du cas d'utilisation « Gérer managers et officiers »}
\label{tab:backlog:ch2:supervision} \\
\hline
\rowcolor{gray!30}
\textbf{Cas d'utilisation} & Gérer managers et officiers \\
\hline
\endfirsthead

\hline
\endhead

\hline
\endfoot

\hline \hline
\endlastfoot

\textbf{Acteur} & Administrateur, Manager, Officier \\
\hline

\textbf{Résumé} &
\begin{itemize}[label=]
  \item \textbf{Scénario 1: Gérer les managers et officiers.}
  \item L'Administrateur gére les managers et officiers, afin de controler les activité internes de la plateforme.
  \item \textbf{Scénario 2:Consulter la liste des managers et officiers.}
  \item L'administrateur peut visualiser tous les managers et officiers inscrits dans le système.
  \item \textbf{Scénario 3 :Filtrer la liste des managers et officiers.}
  \item L'administrateur peut filtrer la liste des managers et officiers par rôle,statut(Pending/Actif/Bloqué) et rechercher par détails.
  \item \textbf{Scénario 4 :Activer les comptes des managers et officiers.}
  \item L'administrateur peut activer les comptes des managers et officiers pour les rendre actifs.
  \item \textbf{Scénario 5 :Bloquer un compte.}
  \item L'administrateur peut bloquer un compte pour le rendre inactif.
  \item \textbf{Scénario 6 :Débloquer un compte.}
  \item L'administrateur peut débloquer un compte pour le rendre actif.
  \item \textbf{Scénario 7 :Supprimer un compte.}
  \item L'administrateur peut supprimer un compte en le bloquant ou le débloquant.
\end{itemize} \\
\hline

\textbf{Pré-conditions} &
\begin{itemize}[label=]
  \item \textbf{Scénario 1 :} Gérer les managers et officiers.
  \item 1.1 L'administrateur à le privilège administratif.
\end{itemize} \\
\hline
\textbf{} &
\begin{itemize}[label=]
\item 1.2 L'administrateur accède au tableau de bord d'administration.
\item 1.3 L'administrateur clique sur « Manage Workers ».


  \item \textbf{Scénario 2 :} Consulter la liste des managers et officiers.
  \item 2.1 Le scénario 1 « Gérer les managers et officiers » est bien exécuté.

  \item \textbf{Scénario 3 :} Activer les comptes des managers et officiers.
  \item 3.1 Le scénario 2 « Consulter la liste des managers et officiers » est bien exécuté.

  \item \textbf{Scénario 4 :} Bloquer un compte.
  \item 4.1 Le scénario 2 « Consulter la liste des managers et officiers » est bien exécuté.

  \item \textbf{Scénario 5 :} Débloquer un compte.
  \item 5.1 Le scénario 2 « Consulter la liste des managers et officiers » est bien exécuté.

  \item \textbf{Scénario 6 :} Supprimer un compte.
  \item 6.1 Le scénario 2 « Consulter la liste des managers et officiers » est bien exécuté.
  \item \textbf{Scénario 7 :} Supprimer un compte.
  \item 7.1 Le scénario 2 « Consulter la liste des managers et officiers » est bien exécuté.

\end{itemize} \\
\hline
\textbf{Description de scénario nominal} &
\begin{itemize}[label=]
  \item \textbf{Scénario 1 :} Gérer les managers et officiers.
  \item 1.1- L'administrateur accède au tableau de bord d'administration avec le privilège administratif.
  \item \textbf{Scénario 2 :} Consulter la liste des managers et officiers.


\end{itemize} \\
\hline
\textbf{}&
\begin{itemize}[label=]
\item 2.1- Le système affiche la liste complète des managers et officiers.
   \item 2.2- Le système affiche les informations : nom, email, rôle, statut, date d'inscription.

  \item \textbf{Scénario 3 :} Filtrer la liste des managers et officiers.
  \item 3.1- L'Administrateur choisit le filtre à appliquer.
  \item 3.2- L'Administrateur clique sur l'action « Filtrer ».
  \item 3.3- Le système affiche la liste filtrée des managers et officiers.

  \item \textbf{Scénario 4 :} Activer les comptes des managers et officiers.
  \item 4.1- Le système affiche la liste des managers et officiers non activés.
  \item 4.2- L'administrateur clique sur l'action « Activer » pour activer un compte.
  \item 4.3- Le système affiche un message de confirmation.
  \item 4.4- L'Administrateur confirme l'action.
  \item 4.5- Le system affiche un message de succès.

  \item \textbf{Scénario 5 :} Bloquer un compte.
  \item 5.1- Le système affiche la liste des managers et officiers actifs.
  \item 5.2- L'administrateur clique sur l'action « Bloquer » pour bloquer un compte.
  \item 5.3- Le système affiche un message de confirmation.
  \item 5.4- L'Administrateur confirme l'action.
  \item 5.5- Le system affiche un message de succès.

  \item \textbf{Scénario 6 :} Débloquer un compte.
  \item 6.1- Le système affiche la liste des managers et officiers bloqués.
  \item 6.2- L'administrateur clique sur l'action « Débloquer » pour débloquer un compte.
  \item 6.3- Le système affiche un message de confirmation.

\end{itemize}\\
\hline
\textbf{} &
\begin{itemize}[label=]
\item 6.4- L'Administrateur confirme l'action.
  \item 6.5- Le system affiche un message de succès.

  \item \textbf{Scénario 7 :} Supprimer un compte.
  \item 7.1- Le système affiche la liste des managers et officiers.
  \item 7.2- L'administrateur clique sur l'action « Supprimer » pour supprimer un compte.
  \item 7.3- Le système affiche un message de confirmation.
  \item 7.4- L'Administrateur confirme l'action.
  \item 7.5- Le system affiche un message de succès.
\end{itemize}\\
\hline

\textbf{Post-conditions} &
\begin{itemize}[label=]
  \item \textbf{Scénario 4 :Activer les comptes des managers et officiers}
  \item 4.1 L'administrateur a activé les comptes des managers et officiers.
  \item \textbf{Scénario 5 :Bloquer un compte}
  \item 5.1 L'administrateur a bloqué les comptes des managers et officiers.
  \item \textbf{Scénario 6 :Débloquer un compte}
  \item 6.1 L'administrateur a débloqué les comptes des managers et officiers.
  \item \textbf{Scénario 7 :Supprimer un compte}
  \item 7.1 L'administrateur a supprimé les comptes des managers et officiers.
\end{itemize} \\
\hline
\end{longtable}

\textbf{ – Raffinement de cas d'utilisation « Gérer profil »}\\
La \hyperref[fig:3.5]{figure 3.5} montre le raffinement de cas d'utilisation « Gérer profil ». L'acteur est capable de mettre à jour ses informations.
\begin{figure}[H]
\centering
\includegraphics[width=\textwidth] {chapitre2/profile/UC Gérer-compte.png}
\caption{ Sprint 1 -Diagramme de cas d'utilisation « Gérer profil »
}
\label{fig:3.4}
\end{figure}

Le Tableau 3.5 représente une description textuelle du cas d'utilisation « Gérer profil ». Il détaille le scénario nominal ainsi que les enchaînements alternatifs.
\begin{longtable}{|>{\arraybackslash}p{4.2cm}|>{\arraybackslash}p{12.5cm}|}
\caption{\centering Description textuelle du sous cas d'utilisation «Gérer Profil»}
\label{tab:backlog:ch2:5} \\
\hline
\rowcolor{gray!30}
\textbf{Cas d'utilisation} & Gérer profil \\
\hline
\endfirsthead

\hline
\endhead

\hline
\endfoot

\hline \hline
\endlastfoot

\textbf{Acteur}  & Citoyen, Manager, Officer \\
\hline
\textbf{Résumé} &
\begin{itemize}[label=]
  \item\textbf{Scénario 1: Consulter Profil}
  \item L'acteur peut consulter son profil complet avec toutes les informations personnelles et professionnelles.
  \item\textbf{Scénario 2: Modifier informations du profil}
  \item L'acteur peut modifier ses informations personnelles (nom, prénom, adresse, date de naissance, numéro d'identité), ses informations professionnelles (titre du poste, lieu de travail) et sa photo de profil.

\end{itemize}\\

 \hline
 \textbf{}&
\begin{itemize}[label=]
  \item\textbf{Scénario 3: Changer le mot de passe}
  \item L'acteur peut changer son mot de passe en fournissant son mot de passe actuel.
  \item\textbf{Scénario 4: Révoquer les sessions actives}
  \item L'acteur peut consulter et révoquer ses sessions actives sur différents appareils.
   \item\textbf{Scénario 5: Supprimer le compte}
  \item L'acteur peut supprimer définitivement son compte.

\end{itemize}\\

\hline
\textbf{Pré-conditions} &
\begin{itemize}[label=]
  \item\textbf{Scénario 1: Consulter Profil}
  \item L'acteur doit être authentifié et doit cliquer sur le bouton "Profil".
  \item\textbf{Scénario 2: Modifier informations du profil}
  \item Le scénario 1 « Consulter Profil » est bien exécuté.
  \item\textbf{Scénario 3: Changer le mot de passe}
   \item Le scénario 1 « Consulter Profil » est bien exécuté.
  \item\textbf{Scénario 4: Révoquer les sessions actives}

\end{itemize}\\

\hline
\textbf{}&
\begin{itemize}[label=]
    \item Le scénario 1 « Consulter Profil » est bien exécuté.
   \item\textbf{Scénario 5: Supprimer le compte}
\item Le scénario 1 « Consulter Profil » est bien exécuté.
\end{itemize}\\
\hline
\textbf{Description de scénario nominal }  &
\begin{itemize}[label=]

  \item\textbf{Scénario 1: Consulter Profil}
    \item{1-} Le système affiche l'interface « Profil » avec les sections suivantes:
    \begin{itemize}
      \item Identité du profil (avec photo)
      \item Informations personnelles (nom, prénom, date de naissance, numéro d'identité)
      \item Informations professionnelles (titre du poste, lieu de travail)
      \item Informations de contact (adresse, ville, pays, code postal)
      \item Paramètres de sécurité (option de changement de mot de passe)
      \item Sessions actives (liste des appareils connectés)
      \item Zone de danger (option de suppression de compte)
    \end{itemize}
    \item{2-} Le système affiche les informations actuelles de l'acteur dans chaque section.
   \item\textbf{Scénario 2: Modifier informations du profil}
   \item{1-} L'acteur peut modifier les champs souhaités dans différentes sections :
   \begin{itemize}
       \item Section "Informations personnelles" : nom, prénom, date de naissance, numéro d'identité, à propos de moi
     \item Section "Informations professionnelles" : titre du poste, lieu de travail
   \end{itemize}

\end{itemize}\\
\hline
\textbf{}&
\begin{itemize}[label=]

     \item Section "Photo de profil" : en cliquant sur l'icône de caméra pour sélectionner une nouvelle image

 \item{2-} L'acteur clique sur le bouton « Enregistrer les modifications » ou sélectionne une nouvelle photo.

\item{3-} Le système vérifie la validité des informations fournies.
     \item{4-} Si le numéro d'identité est modifié, le système vérifie son unicité dans la base de données.
     \item{5-} Si une photo est sélectionnée, le système vérifie le format et la taille de l'image.
    \item{6-} Le système sauvegarde les modifications.
    \item{7-} Le système affiche un message de confirmation "Profil mis à jour avec succès".
\item{8-} Le système met à jour l'affichage avec les nouvelles informations.
    \item\textbf{Scénario 3: Changer le mot de passe}
   \item{1-} L'acteur clique sur le bouton "Changer le mot de passe" dans la section "Paramètres de sécurité".
   \item{2-} Le système affiche un formulaire avec les champs: mot de
\item passe actuel, nouveau mot de passe, confirmer le nouveau mot de passe.
    \item{3-} L'acteur remplit les champs et clique sur "Changer le mot de passe".
    \item{4-} Le système vérifie que le mot de passe actuel est correct.
    \item{5-} Le système vérifie que le nouveau mot de passe respecte les critères de sécurité.
    \item{6-} Le système vérifie que les deux champs de nouveau mot de passe correspondent.



\end{itemize}\\
\hline
\textbf{}&
\begin{itemize}[label=]
 \item{7-} Le système met à jour le mot de passe.
\item{8-} Le système affiche un message de confirmation "Mot de passe changé avec succès".

  \item\textbf{Scénario 4: Révoquer les sessions actives}
   \item{1-} Le système affiche la liste des sessions actives avec les informations suivantes: appareil, type d'appareil, adresse IP, localisation, dernière activité.
    \item{2-} L'acteur peut identifier sa session actuelle (marquée comme "Session actuelle").
    \item{3-} Pour les autres sessions, l'acteur peut cliquer sur "Révoquer la session".
    \item{4-} Le système révoque la session sélectionnée.
    \item{5-} Le système met à jour la liste des sessions actives.

    \item{6-} Le système affiche un message de confirmation "Session révoquée avec succès".
    \item\textbf{Scénario 5: Supprimer le compte}
    \item{1-} L'acteur clique sur le bouton "Supprimer le compte" dans la section "Zone de danger".
    \item{2-} Le système affiche une boîte de dialogue de confirmation avec
\item  un avertissement sur la nature irréversible de cette action.
\item{3-} L'acteur confirme la suppression.
    \item{4-} Le système supprime le compte de l'utilisateur.
    \item{5-} Le système déconnecte l'utilisateur.
    \item{6-} Le système redirige l'utilisateur vers la page d'accueil avec un message "Votre compte a été supprimé avec succès".
\end{itemize}\\
\hline
\textbf{enchaînements Alternatifs} &
\begin{itemize}[label=]
  \item\textbf{Scénario 2: Modifier informations du profil}
    \item{3.1.} Les données saisies sont vides ou non valides: le système affiche un message d'erreur spécifique et ne sauvegarde pas les modifications.
    \item{4.1.} Le numéro d'identité existe déjà: le système affiche un message d'erreur "Ce numéro d'identité est déjà utilisé" et ne sauvegarde pas les modifications.
    \item{5.1.} Le fichier sélectionné n'est pas une image valide: le système affiche un message d'erreur "Veuillez sélectionner une image valide".
    \item{5.2.} L'image dépasse la taille maximale autorisée: le système affiche un message d'erreur "L'image est trop volumineuse. Taille maximale: 5 MB".

  \item\textbf{Scénario 3: Changer le mot de passe}



\end{itemize}\\

\hline
\textbf{}&
\begin{itemize}[label=]
\item

    \item{4.1.} Le "Mot de passe actuel incorrect" et ne change pas le mot de passe.
    \item{5.1.} Le nouveau mot de passe ne respecte pas les critères de sécurité: le système affiche un message d'erreur détaillant les exigences de sécurité.
    \item{6.1.} Les deux champs de nouveau mot de passe ne correspondent pas: le système affiche un message d'erreur "Les mots de passe ne correspondent pas".
    \item\textbf{Scénario 5: Supprimer le compte}
    \item{3.1.} L'acteur annule la suppression: le système ferme la boîte de dialogue et aucune action n'est effectuée.
\end{itemize}\\
\hline
\textbf{Post-conditions } &
\begin{itemize}[label=]
  \item\textbf{Scénario 1: Consulter Profil}
  \item L'acteur visualise toutes ses informations de profil.

  \item\textbf{Scénario 2: Modifier informations du profil}
  \item Les informations personnelles, professionnelles et/ou la photo de profil de l'acteur sont mises à jour dans le système.

  \item\textbf{Scénario 3: Changer le mot de passe}
  \item Le mot de passe de l'acteur est mis à jour dans le système.
  \item\textbf{Scénario 4: Révoquer les sessions actives}

  \item Les sessions révoquées ne sont plus actives et l'utilisateur devra se reconnecter sur ces appareils.

  \item\textbf{Scénario 5: Supprimer le compte}
  \item Le compte de l'acteur est définitivement supprimé du système.
\end{itemize} \\
\end{longtable}




\subsection{Conception}

Dans cette section, nous proposons une étude conceptuelle des données à travers la présentation du diagramme de classes et des diagrammes d'interactions.

\subsubsection{Diagramme de classes}

Le \textit{diagramme de classes} est un outil fondamental permettant de représenter la structure interne d'un système, en exposant les différentes classes, leurs attributs, ainsi que les relations structurelles qui les lient.

La \hyperref[fig:3.6]{Figure~3.6} illustre le diagramme de classes utilisé pour le développement du premier sprint.

\begin{figure}[H]
    \centering
    \includegraphics[width=\linewidth]{chapitre2/sprint1Class.png}
    \caption{Sprint 1~: Diagramme de classes \og Gérer utilisateur \fg{}}
    \label{fig:3.6}
\end{figure}

\subsubsection{Diagrammes d'interaction détaillés}

Dans cette sous-section, nous présentons plusieurs diagrammes de séquence détaillant l'interaction entre la partie \textit{front-end} et la partie \textit{back-end}.

\medskip
\noindent\textbf{\textendash{} Diagramme d'interaction \og Google Auth \fg{}}\\
\hspace{1em}La fonctionnalité \og Google Auth \fg{} permet à un utilisateur de se connecter à la plateforme \textbf{Identity-Secure} en utilisant son compte Google sans avoir à remplir les formulaires, quel que soit le cas d'utilisation, s'inscrire ou s'authentifier (voir la \hyperref[fig:3.7]{Figure~3.7}).
\medskip

\noindent\textbf{\textendash{} Diagramme d'interaction \og S'inscrire \fg{}}\\
\hspace{1em}La fonctionnalité \og S'inscrire \fg{} permet à un internaute de créer un compte sur la plateforme \textbf{Identity-Secure}. L'inscription peut s'effectuer soit via le formulaire standard (voir la \hyperref[fig:3.8]{Figure~3.8}), soit en utilisant un compte Google(voir la \hyperref[fig:3.7]{Figure~3.7}).Via le formulaire en fournissant les informations requises : nom d'utilisateur, adresse e-mail et mot de passe. Une fois l'inscription validée, l'internaute reçoit un e-mail de vérification contenant un lien d'activation. Après validation, le rôle \og Citoyen \fg{} lui est automatiquement attribué, lui donnant accès aux fonctionnalités de la plateforme. Les rôles de \og manager \fg{} et \og officer \fg{} sont réservés aux comptes prédéfinis avec l'approbation de l'administrateur et ne sont pas accessibles lors de l'inscription standard (voir la \hyperref[fig:3.9]{Figure~3.9}).



\medskip

\noindent\textbf{\textendash{} Diagramme d'interaction \og Vérifier mail \fg{}}\\
\hspace{1em}La fonctionnalité \og Vérifier mail \fg{} permet à l'internaute de confirmer son inscription et d'activer son compte sur la plateforme \textbf{Identity-Secure}. Après avoir complété le processus d'inscription, un e-mail de vérification est automatiquement envoyé à l'adresse fournie. Cet e-mail contient un lien d'activation sur lequel l'internaute doit cliquer pour valider son adresse e-mail et finaliser son inscription.

\hspace{1em}En cliquant sur ce lien, l'internaute est redirigé vers une page de confirmation où son compte est marqué comme \og vérifié \fg{}. Un message de succès est alors affiché, et l'internaute peut désormais se connecter à la plateforme. Sans cette vérification, toute tentative de connexion est bloquée par un message invitant à vérifier l'adresse e-mail.

\hspace{1em}Si l'internaute ne reçoit pas l'e-mail de vérification ou si le lien d'activation à été expiré (après 48~heures), il peut demander l'envoi d'un nouveau lien depuis la page de connexion en cliquant sur l'option \og Resend verification email \fg{}. Cette fonctionnalité garantit que seuls les utilisateurs disposant d'une adresse e-mail valide peuvent accéder aux services de la plateforme (voir la \hyperref[fig:3.10]{Figure~3.10}).

\medskip

\noindent\textbf{\textendash{} Diagramme d'interaction \og S'authentifier \fg{}}\\
\hspace{1em}La fonctionnalité \og S'authentifier \fg{} permet les Acteurs existants de se connecter à la plateforme \textbf{Identity-Secure} en saisissant leur nom d'utilisateur et leur mot de passe(voir la \hyperref[fig:3.11]{Figure~3.11}), ou en utilisant leur compte Google(voir la \hyperref[fig:3.7]{Figure~3.7}). Le système vérifie l'identité de l'utilisateur en consultant la base de données et s'assure que l'adresse e-mail a bien été validée.

\hspace{1em}Si l'utilisateur tente de se connecter sans avoir vérifié son adresse e-mail, le système affiche un message d'erreur et propose de renvoyer l'e-mail de validation. Une fois l'authentification réussie, un jeton d'authentification (JWT) est généré, contenant les informations sur le rôle de l'utilisateur (Citoyen, Manager ou Officer) et ses permissions associées. L'utilisateur est redirigé vers le tableau de bord approprié selon son rôle.

\hspace{1em}Ce jeton permet un accès sécurisé aux fonctionnalités correspondant au rôle de l'acteur, avec une redirection automatique vers le tableau de bord approprié~: \textit{citizen-dashboard} pour les utilisateurs standard, \textit{manager-dashboard} pour les managers, et \textit{officer-dashboard} pour les officers. L'option \og Remember me \fg{} permet également à l'acteur de rester connecté sur son appareil pour une période prolongée (voir la \hyperref[fig:3.12]{Figure~3.12}).

\medskip

\noindent\textbf{\textendash{} Diagramme d'interaction \og Récupérer mot de passe \fg{}}\\
\hspace{1em}La fonctionnalité \og Récupérer mot de passe \fg{} permet aux utilisateurs de demander la réinitialisation de leur mot de passe en cas d'oubli. Lorsqu'un utilisateur soumet son adresse e-mail sur la page \og Reset Password \fg{}, le système vérifie l'existence de cette adresse dans la base de données, puis génère un jeton de réinitialisation unique et l'envoie par e-mail.

\hspace{1em}Ce jeton est intégré dans un lien de réinitialisation sécurisé qui expire après une heure. Lorsque l'utilisateur clique sur ce lien, il est redirigé vers la page \og New Password \fg{} où il peut saisir et confirmer son nouveau mot de passe. Le système vérifie alors la correspondance des mots de passe et le respect des critères de sécurité requis (longueur minimale de 8~caractères, présence de lettres majuscules, minuscules, chiffres et caractères spéciaux).

\hspace{1em}Une fois validé, le système hache le nouveau mot de passe pour garantir sa sécurité et met à jour les informations de l'utilisateur dans la base de données. Un message de confirmation \og Your password has been reset successfully. You can now login with your new password. \fg{} est alors affiché, et l'utilisateur est redirigé vers la page de connexion.

\hspace{1em}Cette approche sécurisée garantit que seuls les utilisateurs ayant accès à l'adresse e-mail associée au compte peuvent réinitialiser leur mot de passe, et que le processus est protégé contre les tentatives d'accès non autorisées grâce à l'expiration du jeton et au hachage du mot de passe (voir la \hyperref[fig:3.13]{Figure~3.13}).

\medskip

\noindent\textbf{\textendash{} Diagramme d'interaction \og Gérer Officiers et Managers\fg{}}\\
\hspace{1em}La fonctionnalité \og Gérer Officier et Manager \fg{} permet aux managers et officiers de gérer les comptes des officiers et managers.

\hspace{1em}Cette fonctionnalité est une étape importante pour la gestion des comptes des Officiers et Managers (voir la \hyperref[fig:3.14]{Figure~3.14}).L'administrateur avec un privilège administratif peut consulter la liste de travailleurs (officier et managers) (voir la \hyperref[fig:3.15]{Figure~3.15}) avec la possibilité de filtrage par des critères (voir la \hyperref[fig:3.16]{Figure~3.16}), afin d'activer leurs comptes (voir la \hyperref[fig:3.17]{Figure~3.17}), les bloquer (voir la \hyperref[fig:3.18]{Figure~3.18})(voir la \hyperref[fig:3.19]{Figure~3.19})(voir la \hyperref[fig:3.20]{Figure~3.20}), les débloquer (voir la \hyperref[fig:3.21]{Figure~3.21}) (voir la \hyperref[fig:3.22]{Figure~3.22}) et les supprimer (voir la \hyperref[fig:3.23]{Figure~3.23}).
\medskip

\noindent\textbf{\textendash{} Diagramme d'interaction \og Gérer profil \fg{}}\\
\hspace{1em}La fonctionnalité \og Gérer profil \fg{} est au cœur du système \textbf{Identity-Secure}, permettant aux utilisateurs (Citoyen, Manager, Officer) de gérer l'ensemble de leurs informations personnelles et paramètres de sécurité. Cette fonctionnalité englobe plusieurs sous-fonctionnalités essentielles : la consultation du profil, la modification des informations (personnelles, professionnelles, et photo de profil), le changement de mot de passe, la révocation des sessions actives, et la suppression du compte(voir la \hyperref[fig:3.24]{Figure~3.24}).

\hspace{1em}Lorsqu'un utilisateur accède à son profil, le système récupère toutes ses informations depuis la base de données et les affiche dans une interface organisée en sections distinctes : identité du profil (avec photo), informations personnelles, informations professionnelles, informations de contact, paramètres de sécurité, sessions actives, et zone de danger pour la suppression de compte.(voir la \hyperref[fig:3.25]{Figure~3.25})

\hspace{1em}Dans le scénario de modification des informations, l'acteur saisit les nouvelles données dans le formulaire et clique sur le bouton \og Save Changes \fg{}. Le système procède alors à plusieurs vérifications~:
\begin{itemize}
    \item la validité du format des données saisies (par exemple, format correct pour la date de naissance et le numéro d'identité)
    \item l'unicité du numéro d'identité si celui-ci a été modifié
    \item la cohérence des informations entre elles
\end{itemize}

\hspace{1em}Pour les modifications sensibles comme le numéro d'identité, le système vérifie que cette information n'a pas déjà été définie précédemment, certains champs ne pouvant être modifiés qu'une seule fois(nom d'utilisateur, adresse e-mail) pour des raisons de sécurité.(voir la \hyperref[fig:3.26]{Figure~3.26})

\hspace{1em}Si l'acteur souhaite modifier son mot de passe, il doit cliquer sur l'option dédiée qui ouvre une fenêtre modale. Dans cette fenêtre, il saisit son mot de passe actuel pour authentifier l'opération, puis entre et confirme son nouveau mot de passe. Le système vérifie alors la correspondance avec le mot de passe stocké et le respect des critères de sécurité.(voir la \hyperref[fig:3.27]{Figure~3.27})

\hspace{1em}Pour la révocation des sessions actives, le système affiche la liste des appareils connectés avec leurs informations (type d'appareil, adresse IP, localisation ,etc). L'acteur peut révoquer n'importe quelle session, à l'exception de sa session actuelle.
(voir la \hyperref[fig:3.28]{Figure~3.28})
\hspace{1em}L'utilisateur peut aussi supprimer son compte aprés une confirmation demandé par le system afin de completer cette action critique
(voir la \hyperref[fig:3.29]{Figure~3.29})
\hspace{1em}Cette fonctionnalité offre ainsi une expérience utilisateur complète et sécurisée pour la gestion de l'identité numérique, avec des retours visuels immédiats (messages de confirmation ou d'erreur) pour chaque action effectuée.

% --- Figures ---
% --- S'inscrire global---
\begin{figure}[H]
  \centering
  \includegraphics[width=\linewidth,height=20cm]{chapitre2/inscrire/SD Signup.png}
  \caption{Sprint 1~: Diagramme d'interaction MVC \og S'inscrire \fg{}}
  \label{fig:3.9}
\end{figure}
% --- S'authentifier S'inscrire avec Google ---
\begin{figure}[H]
  \centering
  \includegraphics[width=\linewidth, height=25cm, keepaspectratio]{chapitre2/SD Google Auth.png}
  \caption{Sprint 1~: Diagramme d'interaction MVC \og Google Auth \fg{}}
  \label{fig:3.7}
\end{figure}
\vspace{-10pt}
% --- S'inscrire avec Formulaire ---
\begin{figure}[H]
  \centering
  \includegraphics[width=\linewidth, height=24cm]{chapitre2/inscrire/SD Signup-with-form.png}
  \caption{Sprint 1~: Diagramme d'interaction MVC \og S'inscrire avec formulaire \fg{}}
  \label{fig:3.8}
\end{figure}


% --- vérifier mail ---
\begin{figure}[H]
    \centering
    \includegraphics[width=\linewidth, height=24cm]{chapitre2/inscrire/SD Vérifier mail.png}
    \caption{Sprint 1~: Diagramme d'interaction MVC \og Vérifier mail \fg{}}
    \label{fig:3.8}
\end{figure}
% --- s'authentifier avec formulaire---
\begin{figure}[H]
  \centering
  \includegraphics[width=\textwidth]{chapitre2/auth/SD Signin-with-form.png}
  \caption{Sprint 1~: Diagramme d'interaction MVC \og S'authentifier avec formulaire \fg{}}
  \label{fig:3.11}
\end{figure}
% --- s'authentifier global---
\begin{figure}[H]
    \centering
    \includegraphics[width=\textwidth,height=20cm]{chapitre2/auth/SD Signin.png}
    \caption{Sprint 1~: Diagramme d'interaction MVC \og S'authentifier \fg{}}
    \label{fig:3.12}
\end{figure}
% --- Récupérer mot de passe---
\begin{figure}[H]
    \centering
    \includegraphics[width=\linewidth, height=24cm]{chapitre2/Forgot/SD forget-password.png}
    \caption{Sprint 1~: Diagramme d'interaction MVC \og Récupérer mot de passe \fg{}}
    \label{fig:3.13}
\end{figure}
% --- Gérer profil---
\begin{figure}[H]
    \centering
    \includegraphics[width=\textwidth]{chapitre2/profil/SD Gérer Profile.png}
    \caption{Sprint 1~: Diagramme d'interaction MVC \og Gérer profil \fg{}}
    \label{fig:3.24}
\end{figure}

% --- consulter Profil---
\begin{figure}[H]
  \centering
  \includegraphics[width=\textwidth]{chapitre2/profil/SD Consulter.png}
  \caption{Sprint 1~: Diagramme d'interaction MVC \og Consulter Profil \fg{}}
  \label{fig:3.25}
\end{figure}

% --- Modifier informations---
\begin{figure}[H]
  \centering
  \includegraphics[width=\textwidth,height=23cm]{chapitre2/profil/SD Modifier-informations.png}
  \caption{Sprint 1~: Diagramme d'interaction MVC \og Modifier informations \fg{}}
  \label{fig:3.26}
\end{figure}

% --- Modifier mot de passe---
\begin{figure}[H]
  \centering
  \includegraphics[width=\textwidth,height=24cm]{chapitre2/profil/SD profile-password-change.png}
  \caption{Sprint 1~: Diagramme d'interaction MVC \og Modifier mot de passe \fg{}}
  \label{fig:3.27}
\end{figure}

% --- Révocation des sessions actives---
\begin{figure}[H]
  \centering
  \includegraphics[width=\textwidth]{chapitre2/profil/SD profile-session-management.png}
  \caption{Sprint 1~: Diagramme d'interaction MVC \og Révocation des sessions actives \fg{}}
  \label{fig:3.28}
\end{figure}

% --- Supprimer compte---
\begin{figure}[H]
  \centering
  \includegraphics[width=\textwidth,height=13cm]{chapitre2/profil/SD Supprimer-compte.png}
  \caption{Sprint 1~: Diagramme d'interaction MVC \og Supprimer compte \fg{}}
  \label{fig:3.29}
\end{figure}

%Administrateur
% --- Gérer Officier et Manager---
\begin{figure}[H]
  \centering
  \includegraphics[width=\textwidth,height=12cm]{chapitre2/Admin/SD gererOffMang.png}
  \caption{Sprint 1~: Diagramme d'interaction MVC \og Gérer Officier et Manager \fg{}}
  \label{fig:3.14}
\end{figure}

% --- Consulter Officier et Manager---
\begin{figure}[H]
\centering
\includegraphics[width=\textwidth]{chapitre2/Admin/consulterListe.png}
\caption{Sprint 1~: Diagramme d'interaction MVC \og Consulter Officier et Manager \fg{}}
\label{fig:3.15}
\end{figure}

% --- filtrer liste Officier et Manager---
\begin{figure}[H]
\centering
\includegraphics[width=\textwidth]{chapitre2/Admin/Filtrer Liste.png}
\caption{Sprint 1~: Diagramme d'interaction MVC \og Filtrer liste Officier et Manager \fg{}}
\label{fig:3.16}
\end{figure}

% --- Activer compte Officier et Manager---
\begin{figure}[H]
\centering
\includegraphics[width=\textwidth]{chapitre2/Admin/AcitiverComte.png}
\caption{Sprint 1~: Diagramme d'interaction MVC \og Activer compte \fg{}}
\label{fig:3.17}
\end{figure}

% --- Bloquer compte Officier et Manager---
\begin{figure}[H]
\centering
\includegraphics[width=\textwidth]{chapitre2/Admin/block compte.png}
\caption{Sprint 1~: Diagramme d'interaction MVC \og Bloquer compte \fg{}}
\label{fig:3.18}
\end{figure}

% --- Débloquer compte Officier et Manager---
\begin{figure}[H]
\centering
\includegraphics[width=\textwidth]{chapitre2/Admin/Unblock compte.png}
\caption{Sprint 1~: Diagramme d'interaction MVC \og Débloquer compte \fg{}}
\label{fig:3.19}
\end{figure}

% --- Supprimer compte Officier et Manager---
\begin{figure}[H]
  \centering
  \includegraphics[width=\textwidth]{chapitre2/Admin/supprimer compte.png}
  \caption{Sprint 1~: Diagramme d'interaction MVC \og Supprimer compte \fg{}}
  \label{fig:3.20}
  \end{figure}

% --- End Enhanced Academic Style ---

\clearpage
\subsection{Réalisation}

Dans cette section, nous présentons les interfaces homme-machine développées à l'issue des phases d'analyse et de conception du premier sprint.
\subsubsection*{Page d'accueil}
La page d'accueil d'Identity Secure constitue le point d'entrée principal de la plateforme. Elle présente les fonctionnalités clés du système et guide les utilisateurs vers les actions appropriées selon leur statut (visiteur non authentifié ou utilisateur connecté)\hyperref[fig:3.27]{figure 3.27}.

\paragraph{Composants de la page d'accueil}
La page d'accueil est structurée en plusieurs sections distinctes pour optimiser l'expérience utilisateur :
\begin{itemize}
    \item \textbf{En-tête de navigation} : Menu principal avec liens vers tous les sections de la page d'accueil
    \item \textbf{Section héros} : Présentation du service Identity Secure avec appel à l'action principal
    \item \textbf{Section À propos} : Explication de l'objectif de la plateforme
    \item \textbf{Vitrine des fonctionnalités} : Aperçu des services disponibles (rendez-vous, vérification, sécurité, etc.)
    \item \textbf{Guides d'utilisation} : Instructions spécifiques pour les différents types d'utilisateurs
    \item \textbf{FAQ} : Réponses aux questions fréquemment posées sur le processus Identity Secure
    \item \textbf{Localisation} : Carte interactive des centres de service disponibles
    \item \textbf{Chatbot intégré} : Assistant virtuel pour l'aide en temps réel
    \item \textbf{Pied de page} : Informations de contact et liens utiles
\end{itemize}

\paragraph{Fonctionnalités d'interaction}
La page d'accueil offre plusieurs points d'interaction pour faciliter la navigation :
\begin{itemize}
    \item \textbf{Boutons d'action principaux} : « Commencer » pour l'inscription, « Se connecter » pour l'authentification.
    \item \textbf{Navigation contextuelle} : Liens vers les sections spécifiques selon le profil utilisateur.
    \item \textbf{Recherche de centres} : Localisation géographique des points de service.
    \item \textbf{Chat en direct} : Assistance immédiate via le chatbot intégré.
    \item \textbf{Guides d'utilisation} : Accès aux guides d'utilisation détaillés.
\end{itemize}

\begin{figure}[H]
\centering
\includegraphics[width=0.95\textwidth]{chapitre2/home page.jpeg}
\caption{Sprint 1 – Interface de la page d'accueil}
\label{fig:3.27}

\end{figure}

Pour créer un compte sur Identity Secure, vous disposez de deux méthodes principales d'inscription. La \hyperref[fig:3.28]{figure 3.27} présente l'interface d'inscription standard, où vous devez saisir votre nom d'utilisateur, votre adresse e-mail et un mot de passe. Lors de la saisie, la \hyperref[fig:3.31]{figure 3.31} illustre l'indicateur de force du mot de passe, vous guidant vers la création d'un mot de passe robuste. Si vos mots de passe ne correspondent pas, la \hyperref[fig:3.32]{figure 3.32} vous alerte immédiatement.

En cas de tentative d'utilisation d'un nom d'utilisateur déjà existant, la \hyperref[fig:3.33]{figure 3.33} vous en informe clairement. De même, si votre adresse e-mail est déjà associée à un compte, la \hyperref[fig:3.30]{figure 3.30} vous en avertit. Une fois vos informations validées, la \hyperref[fig:3.29]{figure 3.29} vous guide vers la vérification de votre adresse e-mail , voir la \hyperref[fig:3.34]{figure 3.34}. L'alternative d'inscription via Google est également disponible, comme le montre la \hyperref[fig:3.35]{figure 3.35}, offrant une méthode d'authentification rapide et sécurisée.

\begin{figure}[h!]
\vspace*{0pt}
  \centering

  % Left: tall image
  \begin{minipage}[t]{0.45\textwidth}
    \vspace*{0pt} % ← Forces top alignment
    \centering
    \includegraphics[width=8cm, height=16cm]{chapitre2/Signup/signup-attempt.jpeg}
    \caption{Interface de formulaire d'inscription}
    \label{fig:3.28}
  \end{minipage}%
  \hspace{1cm}
  % Right: shorter image, also top aligned
  \begin{minipage}[t]{0.45\textwidth}
    \vspace*{0pt} % ← Forces top alignment
    \centering
    \includegraphics[width=8cm, height=9cm]{chapitre2/Signup/sign-up-done-wait-for-validation.png}
    \caption{Envoi de l'e-mail de vérification}
    \label{fig:3.29}
  \end{minipage}
\end{figure}
\begin{figure}[H]
  \centering

  % First row
  \begin{minipage}[t]{0.45\textwidth}
    \centering
    \includegraphics[width=8cm, height=6cm]{chapitre2/Signup/Email used.png}
    \caption{Adresse e-mail déjà utilisée}
    \label{fig:3.30}
  \end{minipage}
  \hspace{1cm}
  \begin{minipage}[t]{0.45\textwidth}
    \centering
    \includegraphics[width=8cm, height=6cm]{chapitre2/Signup/Password Indecator.png}
    \caption{Indicateur de force du mot de passe}
    \label{fig:3.31}
  \end{minipage}

\vspace{0.5cm} % force vertical space
\end{figure}



\clearpage
\begin{figure}[H]
  \centering
  % Second row
  \begin{minipage}[t]{0.45\textwidth}
    \centering
    \includegraphics[width=8cm, height=6cm]{chapitre2/Signup/passwords not match.png}
    \caption{Mots de passe non correspondants}
    \label{fig:3.32}
  \end{minipage}
  \hspace{1cm}
  \begin{minipage}[t]{0.45\textwidth}
    \centering
    \includegraphics[width=8cm, height=6cm]{chapitre2/Signup/username used.png}
    \caption{Nom d'utilisateur déjà utilisé}
    \label{fig:3.33}
  \end{minipage}
\end{figure}


% Bloc 3 : Options supplémentaires
\begin{figure}[H]
  \centering
  \begin{minipage}[b]{0.9\textwidth}
    \centering
    \includegraphics[width=\linewidth, height=6cm]{chapitre2/Signup/email-signup-verify.jpg}
    \caption{Boite e-mail}
    \label{fig:3.34}
  \end{minipage}
\end{figure}

\FloatBarrier
\begin{figure}[H]
  \centering
  \begin{minipage}[b]{0.9\textwidth}
    \centering
    \includegraphics[width=\linewidth, height=6cm]{chapitre2/Signup/google auth-interface.jpg}
    \caption{Interface d'inscription via Google}
    \label{fig:3.35}
  \end{minipage}
\end{figure}

\FloatBarrier

\clearpage
\textbf{Interfaces du cas d'utilisation « S'authentifier »}

Pour accéder à votre compte Identity Secure, la \hyperref[fig:3.36]{figure 3.36} présente l'interface de connexion. Vous pouvez vous authentifier en saisissant votre nom d'utilisateur et votre mot de passe, ou utiliser la connexion rapide via Google \hyperref[fig:3.35]{voir figure 3.35} . Si votre compte n'est pas encore vérifié, la \hyperref[fig:3.37]{figure 3.37} vous indique les étapes à suivre, avec l'option de renvoyer un e-mail de vérification comme illustré dans la \hyperref[fig:3.38]{figure 3.38}.

En cas d'erreur lors de la connexion, la \hyperref[fig:3.39]{figure 3.39} montre le message d'échec lié à des identifiants incorrects. Le système vous guide à chaque étape, assurant une expérience de connexion sécurisée et intuitive.

\begin{figure}[h!]
  \centering

  % Left: tall image
  \begin{minipage}[t]{0.45\textwidth}
    \vspace*{0pt} % ← Forces top alignment
    \centering
    \includegraphics[width=8cm, height=16cm]{chapitre2/Signin/signin-attempt.png}
    \caption{Interface d'authentification}
    \label{fig:3.36}
  \end{minipage}%
  \hfill
  % Right:  top aligned
  \begin{minipage}[t]{0.45\textwidth}
    \vspace*{0pt} % ← Forces top alignment
    \centering
    \includegraphics[width=8cm, height=16cm]{chapitre2/Signin/signin not verified.png}
    \caption{Compte non vérifié lors de la connexion}
    \label{fig:3.37}
  \end{minipage}
\end{figure}
\FloatBarrier
\clearpage
\begin{figure}[h!]
  \centering

  % Left: tall image
  \begin{minipage}[t]{0.45\textwidth}
    \vspace*{0pt} % ← Forces top alignment
    \centering
    \includegraphics[width=8cm, height=16cm]{chapitre2/Signin/send verification again.png}
    \caption{Renvoi de l'e-mail de vérification}
    \label{fig:3.38}
  \end{minipage}%
  \hfill
  % Right:  top aligned
  \begin{minipage}[t]{0.45\textwidth}
    \vspace*{0pt} % ← Forces top alignment
    \centering
    \includegraphics[width=8cm, height=16cm]{chapitre2/Signin/signin-failed-for-credentials.png}

    \caption{Échec de connexion - Identifiants incorrects}
    \label{fig:3.39}
  \end{minipage}
\end{figure}
\textbf{Interfaces du cas d'utilisation « Récupérer mot de passe »}

La récupération de mot de passe sur Identity Secure est un processus simple et sécurisé. La \hyperref[fig:3.40]{figure 3.40} présente l'interface initiale où vous saisissez votre adresse e-mail. La \hyperref[fig:3.41]{figure 3.41} vous guide ensuite vers la vérification du code de réinitialisation. Une fois le code vérifié, la \hyperref[fig:3.42]{figure 3.42} vous permet de saisir un nouveau mot de passe.
Un e-mail de réinitialisation est envoyé, comme le montre la \hyperref[fig:3.45]{figure 3.45}, vous permettant de récupérer l'accès à votre compte en toute simplicité.
Si le processus rencontre des difficultés, les\hyperref[fig:3.43] {figures 3.43} et \hyperref[fig:3.44]{3.44} illustrent différents scénarios d'erreur, tels qu'un jeton expiré ou une réinitialisation échouée.

\begin{figure}[H]
  \centering

  % First row
  \begin{minipage}[t]{0.45\textwidth}
    \centering
    \includegraphics[width=8cm, height=10cm]{chapitre2/Forgot password/Reset password attempt.jpeg}
    \caption{Interface de réinitialisation de mot de passe}
    \label{fig:3.40}
  \end{minipage}
  \hspace{1cm}
  \begin{minipage}[t]{0.45\textwidth}
    \centering
    \includegraphics[width=8cm, height=10cm]{chapitre2/Forgot password/Reset password attempt 2.jpeg}
    \caption{Interface de vérification de code de réinitialisation}
    \label{fig:3.41}

  \end{minipage}

  \vspace{0.5cm}
  % Second row
  \begin{minipage}[t]{0.45\textwidth}
    \centering
    \includegraphics[width=8cm, height=10cm]{chapitre2/Forgot password/reset-password-new-password.jpeg}
    \caption{Interface de saisie le nouveau mot de passe}
    \label{fig:3.42}
  \end{minipage}
  \hspace{1cm}
  \begin{minipage}[t]{0.45\textwidth}
    \centering
    \includegraphics[width=8cm, height=10cm]{chapitre2/Forgot password/reset-password-fraud-or-token-passed.jpeg}
    \caption{Jeton expiré ou invalide}
    \label{fig:3.43}
  \end{minipage}


\end{figure}
\begin{figure}[H]
  \centering

  % Left image (tall)
  \begin{minipage}[t]{0.45\textwidth}
    \vspace*{0pt} % Assure top alignment
    \centering
    \includegraphics[width=8cm, height=8cm]{chapitre2/Forgot password/reset-password-failed.jpeg}
    \label{fig:3.44}
    \caption{Échec de réinitialisation de mot de passe}
  \end{minipage}
  \hspace{1cm}
  % Right image (shorter)
  \begin{minipage}[t]{0.45\textwidth}
    \vspace*{0pt} % Assure top alignment
    \centering
    \includegraphics[width=8cm, height=3cm]{chapitre2/Forgot password/email-reset-password.jpeg}
    \caption{E-mail de réinitialisation envoyé}
    \label{fig:3.45}
  \end{minipage}

\end{figure}


\textbf{Interfaces du cas d'utilisation « Gérer profil »}

La gestion de votre profil sur Identity Secure est conçue pour être intuitive et complète. La \hyperref[fig:3.41]{figure 3.48} présente l'interface de profil détaillée, où vous pouvez consulter et modifier vos informations personnelles. Lorsque vous effectuez des modifications, la \hyperref[fig:3.41]{figure 3.46} montre le message de confirmation de mise à jour.

La modification de mot de passe, illustrée par la \hyperref[fig:3.41]{figure 3.49}, vous permet de changer votre mot de passe de manière sécurisée, voir \hyperref[fig:3.41]{figure 3.47} pour le cas de succès. Pour les paramètres avancés, la \hyperref[fig:3.41]{figure 3.50} offre une vue sur la révocation des sessions actives et l'option de suppression de compte, vous donnant un contrôle total sur votre identité numérique.

\begin{figure}[H]
  \centering
  % Block of 3 stacked images (full width)
  \begin{minipage}[t]{\textwidth}
    \centering
    \includegraphics[width=0.5\textwidth, height=1.8cm]{chapitre2/profile/profile-save -changes.png}
    \caption{Confirmation de sauvegarde des modifications}
    \label{fig:3.46}
    \vspace{0.3cm}
    \includegraphics[width=0.5\textwidth, height=1.8cm]{chapitre2/profile/Profile changing password success.png}
    \caption{Changement de mot de passe réussi}
    \label{fig:3.47}
  \end{minipage}

\end{figure}
\begin{figure}[H]
  \centering

  % Large image alone (full width)
  \begin{minipage}[t]{\textwidth}
    \centering
    \includegraphics[width=\textwidth, height=24cm]{chapitre2/profile/profile interface.jpeg}
    \caption{Interface du profil utilisateur}
    \label{fig:3.48}
  \end{minipage}
\end{figure}
\begin{figure}[H]
  \centering
  % Block of 3 stacked images (full width)
  \begin{minipage}[t]{\textwidth}
    \centering
    \includegraphics[width=\textwidth, height=10cm]{chapitre2/profile/profile change password part.jpg}
    \caption{Interface de changement de mot de passe}
    \label{fig:3.49}

    \vspace{0.3cm}

    \includegraphics[width=\textwidth, height=5.5cm]{chapitre2/profile/profile-delete session.png}
    \caption{Révocation des sessions actives}
    \label{fig:3.50}
    \vspace{0.3cm}

    \includegraphics[width=\textwidth, height=6cm]{chapitre2/profile/profile-delete account .png}
    \caption{Suppression du compte}
    \label{fig:3.51}
  \end{minipage}

\end{figure}
\textbf{Interfaces du cas d'utilisation « Gérer Officiers et Managers»}

L’interface dédiée à la gestion des managers et officiers permet à l’administrateur de consulter la liste des travailleurs, d’appliquer des filtres (rôle, statut, recherche), et d’effectuer des actions telles qu’activer, bloquer, débloquer ou supprimer un compte. Chaque opération est accessible via des boutons d’action et confirmée par une boîte de dialogue, garantissant la sécurité des modifications. La \hyperref[fig:3.52]{figurer 3.52} illustre cette interface d’administration centralisée.
\begin{figure}[H]
\centering
\includegraphics[width=\linewidth]{chapitre2/AdminPortal.png}
    \caption{Interface Administrateur}
    \label{fig:3.51}

\end{figure}


\addcontentsline{toc}{section}{Conclusion}
Au terme de ce premier sprint, les principales fonctionnalités de gestion des utilisateurs et des rôles ont été conçues et réalisées. Les utilisateurs peuvent désormais s’inscrire, s’authentifier, récupérer leur mot de passe, gérer leur profil, et l’administrateur peut superviser les comptes des managers et officiers. Cette base solide permettra d’étendre le système vers de nouveaux modules, comme la gestion des rendez-vous, présentée dans le chapitre suivant.
\label{sec_Conclusion}
