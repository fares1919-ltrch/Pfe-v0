
\chapter{Release 2 :« Gérer Rendez-vous » }
\label{chap_sprint3}
\addcontentsline{toc}{section}{Introduction}

\section*{sprint 3 :« Gérer Rendez-vous » }
 Dans cette partie, nous traitons les points suivants : tout d'abord, l'organisation et le sprint backlog. Ensuite, nous décrivons le contenu du deuxiéme sprint, intitulé "Gérer rendez-vous". Nous entamons ensuite la phase d'analyse et nous explorons les solutions conceptuelles. Enfin, nous présentons les différentes réalisations obtenues.
\subsection{Organisation}
Le tableau 4.1 ci-dessous donne un aperçu détaillé sur le Backlog du troisième sprint qui
prend en charge une fonctionnalité "Gérer Rendez-vous".
\\

\begin{longtable}{|>{\centering\arraybackslash}p{0.7cm}|>{\arraybackslash}p{5cm}|>{\centering\arraybackslash}p{1.3cm}|>{\arraybackslash}p{6.5cm}|>{\centering\arraybackslash}p{1cm}|}
\caption{\centering Backlog du sprint 3 : « Gérer Rendez-vous»}
\label{tab:backlog} \\

\hline
\rowcolor{gray!30}
ID & User Story & ID & Tâche & Durée /j \\
\hline
\endfirsthead


\hline
\endhead

\hline
\endfoot

\hline
\endlastfoot

\multirow{3}{0.7cm}{5.1} & \multirow{3}{3.8cm}{En tant que citoyen, je peux soumettre un rendez-vous en choisissant un centre  et une date selon les créneaux disponibles .} & 5.1.1 & Créer les interfaces pour la sélection d'un centre et d’un créneau à partir du calendrier. & 2 \\
\cline{3-5}
& & 5.1.2 & Développer les API pour la gestion des rendez-vous avec choix d’un créneau disponible. & 1 \\
\cline{3-5}
& & 5.1.3 & Tester les fonctionnalités de soumission de rendez-vous. & 1 \\
\hline

\multirow{1}{0.7cm}{5.2} & \multirow{1}{4cm}{En tant que citoyen} & 5.2.1 & Créer l'interface de consultation . & 1 \\
\cline{3-5}

\multirow{2}{0.7cm}{} & \multirow{2}{4cm}
{je peux consulter les détails de mon rendez-vous }  & 5.2.2 & Développer les API pour la consultation. & 1 \\
\cline{3-5}
&  & 5.2.3 & Tester les fonctionnalités de consultation . & 1 \\
\hline

\multirow{3}{0.7cm}{5.3} & \multirow{3}{4cm}{En tant qu’officier, je peux planifier les rendez-vous en les validant ou les rejetant .} & 5.3.1 &  Créer  les  interfaces  pour la  modification du statut d'un rendez-vous. & 1\\
\cline{3-5}
& & 5.3.2 & Développer les API pour modifier le statut un rendez-vous. &1\\
\cline{3-5}
& & 5.3.3 & Tester les fonctionnalités de modification des rendez-vous. & 1 \\
\hline

\end{longtable}


\subsection{Analyse}
Durant cette phase d’analyse, nous approfondissons les diverses fonctionnalités en les accompagnant de leurs cas d’utilisation respectifs


\subsubsection{– Raffinement de cas d’utilisation « Gérer rendez-vous  »
}
La Figure 4.1 montre le raffinement de cas d’utilisation « Gérer rendez-vous  » mettant en lumière la capacité du citoyen et de l'officier de police à gérer les rendez-vous.
\begin{figure}[H]
\centering
\includegraphics[width= 17 cm , height = 6 cm]{assetsChap3/GererDemandeUC19_05.drawio (1)_cropped.pdf}
\caption{ Sprint 3 -Diagramme de cas d’utilisation « Gérer rendez-vous » }
\end{figure}

Le Tableau 4.2 représente une description textuelle du cas d’utilisation « Gérer rendez-vous». Il
détaille le scénario nominal ainsi que les enchaînements alternatifs.


\begin{longtable}{|>{\arraybackslash}p{4.2cm}|>{\arraybackslash}p{12.5cm}|}
\caption{\centering Description textuelle du cas d'utilisation « Gérer rendez-vous »}
\label{tab:backlog} \\
\hline
\rowcolor{gray!30}
\textbf{Cas d'utilisation} & Gérer Rendez-vous \\
\hline
\endfirsthead

\hline
\endhead

\hline
\endfoot

\hline \hline
\endlastfoot
\textbf{Acteur}  & Citoyen et Officer de police\\
\hline
\textbf{Résumé} &
\begin{itemize}[label=]
\item\textbf{Scénario 1: Soumettre rendez-vous  }
  \item Le citoyen brésilien peut soumettre un rendez-vous .
  \item\textbf{Scénario 2: Consulter rendez-vous }
  \item Le citoyen brésilien peut consulter les détails son rendez-vous .
  \item\textbf{Scénario 3: Supprimer rendez-vous }
  \item Le citoyen brésilien peut supprimer son rendez-vous .
  \item\textbf{Scénario 4: Changer date rendez-vous}
  \item Le citoyen peut changer la date de son rendez-vous rejeté.
  \item\textbf{Scénario 5: Plannifier rendez-vous }
  \item L'officier de police peut plannifier un rendez-vous.
  \item\textbf{Scénario 6: Consulter liste des rendez-vous}
  \item L'officier de police peut consulter la liste des rendez-vous.
  \item\textbf{Scénario 7: Valider rendez-vous }
  \item L'officier de police peut valider un rendez-vous .
  \item\textbf{Scénario 8: Rejeter rendez-vous}
  \item L'officier de police peut rejeter un rendez-vous .
\end{itemize} \\
\hline
\textbf{Pré-conditions} &
\begin{itemize}[label=]
\item\textbf{Scénario 1:  Soumettre rendez-vous }
 \item  Le citoyen brésilien est authentifié.
 \item  Le citoyen brésilien clique sur "CPF Request".
\end{itemize}\\



\hline
\textbf{} &
\begin{itemize}[label=]
 \item\textbf{Scénario 2: Consulter rendez-vous }
 \item  Le citoyen brésilien est authentifié.
 \item  Le citoyen brésilien clique sur "My Appointment".
 \item\textbf{Scénario 3: Supprimer rendez-vous }
 \item   Le scénario 2 «Consulter rendez-vous» est bien executé.
 \item\textbf{Scénario 4: Changer date rendez-vous}
 \item Le scénario 2 «Consulter rendez-vous» est bien executé.
 \item le rendez-vous doit avoir le statut "rejected"
 \item\textbf{Scénario 5: Plannifier liste des rendez-vous }
 \item L'officier de police est authentifié.
 \item L'officier de police clique sur 'Requests'.
 \item Un rendez-vous existe déja dans la base.
 \item\textbf{Scénario 6: Consulter rendez-vous }
 \item Le scénario 5 «Plannifier Liste des rendez-vous» est bien executé.
 \item\textbf{Scénario 7: Valider rendez-vous }
 \item   Le scénario 6 «Consulter Liste des rendez-vous» est bien executé.
 \item\textbf{Scénario 8: Rejeter rendez-vous}
 \item Le scénario 6 «Consulter Liste des rendez-vous» est bien executé.
\end{itemize} \\
\hline
\textbf{Description des scénarios nominaux} &
\begin{itemize}[label=]
\item \textbf{Scénario 1 : Soumettre rendez-vous}
        \item 1- Le système demande l'autorisation de localisation pour identifier le centre le plus proche.
        \item 2- Le citoyen autorise sa localisation et modifie sa position si désiré pour choisir le centre convenable.
        \item 3- Le système affiche la nouvelle localisation.
\end{itemize}\\




\hline
\textbf{} &
    \begin{itemize}[label=]
    \item 4- Le citoyen clique sur "choose date".
    \item 5- Le système affiche et charge le calendrier avec les disponibilités de chaque jour.
    \item 6- le citoyen choisit la date et clique sur "submit Request"
    \item 7- Le système enregistre la demande dans la base
    \item\textbf{Scénario 2: Consulter rendez-vous }
    \item 1- Le citoyen clique sur 'My Appointment'
    \item 2- Le système affiche l'interface avec les details du rendez-vous.
    \item \textbf{Scénario 3 : Supprimer rendez-vous}
    \item 1- Le citoyen brésilien clique sur l'icon de suppression de son rendez-vous
    \item 2- Le système demande une confirmation.
    \item 3- Le citoyen brésilien confirme la suppression.
    \item 4- Le système supprime le rendez-vous de la base
    \item \textbf{Scénario 4 : Changer date rendez-vous}
    \item 1- Le citoyen clique sur l'icon du calendrier.
    \item 2- Le système affiche et charge le calendrier avec les disponibilités de chaque jour.
    \item 3- Le citoyen choisit la date , clique sur "apply" et puis sur "reschedule".
    \item 4- Le système met à jour la nouvelle date dans la base
    \item \textbf{Scénario 5 : Plannifier liste des rendez-vous}
    \item 1- L'officier de police clique sur "Requests"
    \item 2- le système affiche l'interface des rendez-vous
    \item\textbf{Scénario 6: Consulter rendez-vous }
    \item 1- Le système affiche la liste des rendez-vous et leurs détails.
\end{itemize} \\






\hline
\textbf{}&
\begin{itemize}[label=]
    \item\textbf{Scénario 7: Valider rendez-vous }
    \item 1- L'officier de police clique sur le boutton "Validate".
    \item 2-Le système met le statut du rendez-vous à jour et le passe à "validated".
    \item\textbf{Scénario 8: Rejeter rendez-vous }
    \item 1- L'officier de police clique sur le boutton "Reject".
    \item 2- Le système demande une confirmation.
    \item 3- L'officier de police confirme.
    \item 4- Le système met le statut du rendez-vous à jour et le passe à "rejected".
\end{itemize}\\
\hline
\textbf{Enchaînements Alternatifs} &
\begin{itemize}[label=]
 \item\textbf{Scénario 1: Soumettre rendez-vous}
 \item L'utilisateur n'autorise pas sa localisation actuelle :démarre au point 2 du scénario nominal.
  \item\textbf{Scénario 3: Supprimer rendez-vous}
  \item le citoyen brésilien annule la suppression
\end{itemize} \\
\hline
\textbf{Post-conditions} &
\begin{itemize}[label=]
\item\textbf{Scénario 1: Soumettre rendez-vous}
\item Une nouvelle rendez-vous est créée dans le système.
\item\textbf{Scénario 2: Consulter rendez-vous}
\item L'interface affiche les détails du rendez-vous.
\item\textbf{Scénario 3: Supprimer rendez-vous }
\item Le rendez-vous est supprimé de la base de données.
\item\textbf{Scénario 4: Changer Date rendez-vous }
\item un rendez-vous est met à jour dans la base de données
\end{itemize} \\





\hline
\textbf{}&
\begin{itemize}[label=]

\item\textbf{Scénario 5: Plannifier rendez-vous}
\item L'interface des rendez-vous est affichée.
\item\textbf{Scénario 6: Consulter liste des rendez-vous}
\item La liste des rendez-vous est affichée avec leurs détails.
\item\textbf{Scénario 7: Valider Date rendez-vous }
\item un rendez-vous est met à jour dans la base de données
\item\textbf{Scénario 8: Rejeter Date rendez-vous }
\item un rendez-vous est met à jour dans la base de données
\end{itemize}\\
\end{longtable}


\subsection{Conception}
Dans cette section, nous présentons l’étude conceptuelle des données par la présentation du
diagramme de classes et des diagrammes d’interactions.
\subsubsection{Diagramme de classes}
Le diagramme de classes est un outil permettant de représenter la structure interne d’un
système en exposant les différentes classes, leurs attributs, ainsi que les relations structurelles
qui les lient.
La figure 4.3 décrit le diagramme de classes que nous avons utilisé pour développer le deuxième
sprint du Release 2.
\begin{figure}[H]
\centering
\includegraphics[width=16 cm, height= 25cm]{assetsChap3/Sprint2rendez-vousClass.png}
\caption{ Diagramme de classes « Gérer rendez-vous  » }
\end{figure}

\subsubsection{Diagrammes d’interaction détaillés}
Dans cette sous-section, nous présentons les diagrammes de séquences de chaque cas d'utilisation qui détaillent
l'interaction entre la partie front-end et la partie back-end du troisième sprint.
\\
\begin{figure}[H]
\centering
\includegraphics[scale=0.82 , width=15 cm, height= 20cm]{assetsChap3/seqSoumettreRenParCitoye.drawio_cropped.pdf}
\caption{ Sprint 2 -Diagramme d’interaction « Soumettre Rendez-vous » }
\end{figure}
\begin{figure}[H]
\centering
\includegraphics[scale=0.83, width=15 cm, height=11 cm]{assetsChap3/seqConsulterMONRendezvous.drawio_cropped.pdf}
\caption{ Sprint 2 -Diagramme d’interaction « Consulter Rendez-vous » }
\end{figure}

\begin{figure}[H]
\centering
\includegraphics[scale=0.82, width=15 cm, height=11 cm]{assetsChap3/seqChanDatee.drawio_cropped.pdf}
\caption{ Sprint 3 -Diagramme d’interaction « Changer Date Rendez-vous » }
\end{figure}

\begin{figure}[H]
\centering
\includegraphics[scale=0.92, width=15 cm, height=11 cm]{assetsChap3/seqSupprimerRendez.drawio_cropped.pdf}
\caption{ Sprint 2 -Diagramme d’interaction « Supprimer Rendez-vous » }
\end{figure}


\begin{figure}[H]
\centering
\includegraphics[scale=0.82, width=15 cm, height=11 cm]{assetsChap3/sePlannifier.drawio (1)_cropped.pdf}
\caption{ Sprint 2 -Diagramme d’interaction « Plannifier Rendez-vous » }
\end{figure}


\begin{figure}[H]
\centering
\includegraphics[scale=0.82, width=15 cm, height=11 cm]{assetsChap3/seqConsulterListRendez.drawio_cropped.pdf}
\caption{ Sprint 2 -Diagramme d’interaction « Consulter Rendez-vous » }
\end{figure}

\begin{figure}[H]
\centering
\includegraphics[scale=0.82, width=15 cm, height=11 cm]{assetsChap3/Copie de seValidateParOfficer.drawio_cropped (1).pdf}
\caption{ Sprint 2 -Diagramme d’interaction « Valider Rendez-vous » }
\end{figure}

\begin{figure}[H]
\centering
\includegraphics[scale=0.99, width=15 cm, height=11 cm]{assetsChap3/seqReject.drawio_cropped.pdf}
\caption{ Sprint 2 -Diagramme d’interaction « Rejeter Rendez-vous » }
\end{figure}




\subsection{Réalisation}
Après avoir analysé le sprint 4 et achevé la phase de conception, cette section présente les interfaces homme-machine élaborées durant ce sprint.\\


\textbf{– Interfaces du cas d'utilisation « soumettre  Rendez-vous »}\\
L'interface de visualisations des différentes demandes à valider ou rejeter.
\begin{figure}[H]
\centering
\includegraphics[width = 15 cm]{assetsChap3/realisation/soumettre1.png}
\caption{ Sprint 2 - Interface « Soumettre Rendez-vous » }
\end{figure}

\begin{figure}[H]
\centering
\includegraphics[width = 15 cm]{assetsChap3/realisation/soumettre2.png}
\caption{ Sprint 2 - Interface « Soumettre Rendez-vous » }
\end{figure}

\begin{figure}[H]
\centering
\includegraphics[width = 15 cm]{assetsChap3/realisation/soumettre3.jpeg}
\caption{ Sprint 2 - Interface « Soumettre Rendez-vous » }
\end{figure}

\begin{figure}[H]
\centering
\includegraphics[width = 15 cm]{assetsChap3/realisation/soumettre4.jpeg}
\caption{ Sprint 2 - Interface « Soumettre Rendez-vous » }
\end{figure}

\begin{figure}[H]
\centering
\includegraphics[width = 15 cm , height= 13 cm]{assetsChap3/realisation/soumettre5.png}
\caption{ Sprint 2 - Interface « Soumettre Rendez-vous » }
\end{figure}













\textbf{– Interfaces du cas d'utilisation « Consulter Liste Rendez-vous »}\\
L'interface de visualisations des différentes demandes à valider ou rejeter.
\begin{figure}[H]
\centering
\includegraphics[width = 15 cm]{assetsChap3/realisation/consulter.png}
\caption{ Sprint 2 - Interface « Consulter Liste Rendez-vous » }
\end{figure}




\textbf{– Interfaces du cas d'utilisation « Rejeter Rendez-vous »}\\
message aprés rejet d'un rendez vous qui est fait à partir de l'interface de consultation en cliquant à reject bouton .
\begin{figure}[H]
\centering
\includegraphics[scale=0.8]{assetsChap3/realisation/reject.png}
\caption{ Sprint 2 - Interface « Consulter Liste Rendez-vous » }
\end{figure}



\textbf{–  cas d'utilisation « valider Rendez-vous »}\\
message aprés validation d'un rendez vous qui est fait à partir de l'interface de consultation en cliquant à validate bouton .
\begin{figure}[H]
\centering
\includegraphics[scale=0.8]{assetsChap3/realisation/validate.png}
\caption{ Sprint 2 - Interface « valider Rendez-vous » }
\end{figure}


\textbf{–  cas d'utilisation « filtrer Rendez-vous »}\\
interface de filtrage selon plusieurs critéres "validate,pending,rejected,all" mais ici on a choisit le critére "validated".
\begin{figure}[H]
\centering
\includegraphics[width = 18 cm]{assetsChap3/realisation/filtrer.png}
\caption{ Sprint 2 - Interface « Filtrer Liste Rendez-vous » }
\end{figure}



\textbf{– Interfaces du cas d'utilisation « consulter Mon Rendez-vous »}\\
interface de consultation de statut d'un rendez-vous soumis.
\begin{figure}[H]
\centering
\includegraphics[width = 18 cm ,height = 10 cm]{assetsChap3/realisation/consulterMonRendezvous.jpeg}
\caption{ Sprint 2 - Interface « Consulter Mon Rendez-vous » }
\end{figure}




\textbf{– Interfaces du cas d'utilisation « Consulter Liste Rendez-vous »}\\
L'interface de collecte biométrique permet à l'officier de police de capturer les différentes données biométriques du citoyen de manière structurée et sécurisée. L'interface inclut des outils de capture pour la photo du visage, les empreintes digitales et les scans d'iris, ainsi que des indicateurs de qualité pour s'assurer que les données capturées répondent aux normes requises.

\begin{figure}[H]
\centering
\includegraphics[width = 17 cm]{assetsChap3/realisation/consulter.png}
\caption{ Sprint 2 - Interface « Consulter Liste Rendez-vous » }
\end{figure}




\addcontentsline{toc}{section}{Conclusion}
\section*{Conclusion}
Durant cette release, nous avons mené à bien l'analyse, la conception et la mise en œuvre du deuxième sprint, à savoir "Gérer rendez-vous ". À ce stade, les utilisateurs ont la possibilité de créer des rendez-vous et de les gérer. Les citoyens peuvent soumettre, consulter, modifier et supprimer leurs rendez-vous, tandis que les officiers peuvent valider ou rejeter les demandes.
\\ Le prochain chapitre introduira la Troisième release du projet.
\label{sec_Conclusion}



