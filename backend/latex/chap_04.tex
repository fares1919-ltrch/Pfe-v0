\chapter{Release 3 :« Gérer CPF \& transactions et fraudes» }
\label{chap_sprint3}
\addcontentsline{toc}{section}{Introduction}
\section*{Introduction}
\label{sec_introduction:ch4}
Après avoir clôturé la deuxième release dédiée à la gestion des rendez-vous, nous allons dans cette section examiner en détail la troisième release du projet, centrée sur la gestion des CPF (Cadastro de Pessoas Físicas). Cette release est cruciale car elle implémente le cœur du système d'identité brésilien, permettant la collecte des données biométriques, la vérification de l'unicité des identités, et la génération sécurisée des numéros CPF.

Dans ce chapitre, nous décrivons l'organisation et le backlog du sprint « Gérer CPF », puis nous abordons la phase d'analyse et la solution conceptuelle en illustrant les interactions entre les acteurs et le système à l'aide de divers diagrammes. Nous détaillons particulièrement le processus automatisé de déduplication biométrique et de génération de CPF qui s'exécute après la collecte des données biométriques lors des rendez-vous. Enfin, nous mettons en lumière la mise en œuvre concrète à travers les interfaces utilisateur pour les citoyens, les officiers et les managers.

\subsection {Sprint 3 : « Gérer CPF »}
Dans cette section, nous décrivons le développement du système de gestion des CPF, qui constitue le cœur de notre application d'identité. Ce système permet la collecte sécurisée des données biométriques, la vérification automatique de l'unicité des identités via un processus de déduplication, et la génération des numéros CPF pour les citoyens brésiliens vérifiés.
\subsubsection{Organisation}
Le tableau 5.1 ci-dessous donne un aperçu détaillé sur le Backlog du sprint « Gérer CPF ».

\begin{longtable}{|>{\centering\arraybackslash}p{0.7cm}|>{\arraybackslash}p{5cm}|>{\centering\arraybackslash}p{1.3cm}|>{\arraybackslash}p{6.5cm}|>{\centering\arraybackslash}p{1cm}|}
\caption{\centering Backlog du sprint 3 : « Gérer CPF »}
\label{tab:backlog} \\

\hline
\rowcolor{gray!30}
ID & User Story & ID & Tâche & Durée /j \\
\hline
\endfirsthead

\hline
\endhead

\hline
\endfoot

\hline
\endlastfoot

\multirow{3}{0.7cm}{7.1} & \multirow{3}{4cm}{En tant qu'officier de police, je peux téléverser et enregistrer les données biométriques des citoyens lors des rendez-vous.} & 7.1.1 & Créer les interfaces pour le téléversement sécurisé des données biométriques. & 1 \\
\cline{3-5}
& & 7.1.2 & Développer les API pour recevoir et stocker les données biométriques avec chiffrement. & 1 \\
\cline{3-5}
& & 7.1.3 & Tester les fonctionnalités de capture et téléversement des données biométriques. & 1 \\
\hline

\multirow{3}{0.7cm}{7.3} & \multirow{3}{4cm}{En tant que manager, je peux consulter les résultats de déduplication détectés dans la base.} & 7.3.1 & Créer les interfaces pour afficher les résultats de déduplication avec filtres. & 1 \\
\cline{3-5}
& & 7.3.2 & Développer les API pour récupérer les résultats de déduplication avec traçabilité. & 1 \\
\cline{3-5}
& & 7.3.3 & Tester les fonctionnalités de consultation des déduplications. & 1 \\
\hline

\multirow{3}{0.7cm}{6.4} & \multirow{3}{4cm}{En tant que citoyen, je peux consulter les détails de mon CPF, et télécharger mon certificat CPF.} & 6.4.1 & Créer les interfaces pour afficher le numéro CPF et son état de génération (en attente, généré, frauduleux) avec possibilité de masquage/affichage. & 1 \\
\cline{3-5}
& & 6.4.2 & Développer les API pour récupérer le CPF et son état de génération (en attente, généré, frauduleux) pour le citoyen authentifié. & 1 \\
\cline{3-5}
& & 6.4.3 & Tester les fonctionnalités de consultation et téléchargement du CPF. & 1 \\
\hline

\multirow{3}{0.7cm}{6.5} & \multirow{3}{5cm}{En tant qu’officier de police, je peux consulter la liste des rendez-vous plannifiées.}
& 6.5.1 & Créer les interfaces pour consulter la liste des citoyens avec filtres avancés. & 1 \\
\cline{3-5}
& & 6.5.2 & Développer les API pour récupérer la liste des citoyens avec filtres. & 1 \\
\cline{3-5}
& & 6.5.3 & Tester les fonctionnalités de consultation et filtrage des citoyens. & 1 \\
\hline

\end{longtable}

\subsubsection{Analyse}
Durant cette phase d'analyse, nous approfondissons les diverses fonctionnalités en les accompagnant de leurs cas d'utilisation respectifs.

\subsubsection{– Raffinement de cas d'utilisation « Gérer CPF »}\\
La Figure 5.1 montre le raffinement de cas d'utilisation « Gérer CPF » mettant en lumière
les interactions entre les citoyens, les officiers et les managers dans le processus de gestion du CPF.

\begin{figure}[H]
\centering
\includegraphics[width = 15 cm  ]{assetschap4/gererCPFUC.drawio_cropped.pdf}
\caption{ Sprint 3 - Diagramme de cas d'utilisation « Gérer CPF ». }
\end{figure}

Le Tableau 5.2 représente une description textuelle du cas d'utilisation « Gérer CPF ». Il
détaille le scénario nominal ainsi que les enchaînements alternatifs.

\begin{longtable}{|>{\arraybackslash}p{4.2cm}|>{\arraybackslash}p{12.5cm}|}
\caption{\centering Description textuelle du cas d'utilisation « Gérer CPF »}
\label{tab:backlog} \\
\hline
\rowcolor{gray!30}
\textbf{Cas d'utilisation} & Gérer CPF \\
\hline
\endfirsthead

\hline
\endhead

\hline
\endfoot

\hline \hline
\endlastfoot
 \textbf{Acteur}  & Citoyen, Officier de Police, Manager CPF\\
\hline
\textbf{Résumé} &
\begin{itemize}[label=]
  \item\textbf{Scénario 1: Consulter CPF}
  \item Le citoyen brésilien peut consulter les détails de son numéro CPF.
  \item\textbf{Scénario 2: Télécharger CPF}
  \item Le citoyen brésilien peut télécharger son certificat CPF.
  \item\textbf{Scénario 3: Consulter Rendez-vous Plannifiées}
  \item L'officier de police peut consulter les rendez-vous plannfiiées .
  \item\textbf{Scénario 4: Téléverser données biométriques}
  \item L'officier de police peut téleverser les données biométriques des citoyens.
\end{itemize}\\
\hline





\textbf{}&
\begin{itemize}[label=]
 \item\textbf{Scénario 5: Consulter Déduplications}
  \item Le manager peut consulter les déduplications et leurs détails.
  \item\textbf{Scénario 6: Filtrer Déduplications CPF}
  \item Le manager peut filtrer les déduplications selon plusieurs critéres.
\end{itemize}
\hline
\textbf{Pré-conditions} &
\begin{itemize}[label=]
  \item\textbf{Scénario 1: Consulter CPF}
  \item Le citoyen brésilien doit etre authetifié.
  \item\textbf{Scénario 2: Télécharger CPF}
  \item Le scénario 1  « Consulter CPF »est bien exécuté .
  \item\textbf{Scénario 3: Consulter Rendez-vous Plannifiées}
  \item L'officier de police doit etre authetifié.
  \item\textbf{Scénario 4: Téléverser données biométriques}
  \item Le scénario  « Consulter Rendez-vous Plannifiées» est bien exécuté.
  \item\textbf{Scénario 5: Consulter Déduplications}
  \item Le manager doit étre authentifié.
  \item\textbf{Scénario 6: Filtrer Déduplications CPF}
  \item Le scénario  « Consulter déduplications» est bien exécuté.
 \end{itemize} \\
\hline
\textbf{Description de scénario nominal }  &
\begin{itemize}[label=]
 \item\textbf{Scénario 1: Consulter CPF}
  \item Le system charge et affiche l'interface de CPF .
  \item\textbf{Scénario 2: Télécharger CPF}
  \item 1- Le citoyen brésilien clique sur le bouton "download CPF information".
\end{itemize} \\










\hline
\textbf{}&
\begin{itemize}[label=]
 \item 2- Le system télécharge la certificat CPF
  \item 3- Le certificat est enregistré dans l'appareil personnel du citoyen
 \item\textbf{Scénario 3: Consulter Rendez-vous Plannifiées}
  \item 1- L'officier de police clique sur Appointement.
  \item 2- Le system charge et affiche l'interface avec les rendez-vous plannifiées
  \item\textbf{Scénario 4: Téléverser données biométriques}
  \item 1- L'officier de police choisit le rendez-vous désiré et clique sur upload
  \item 2- Le system charge et affiche l'interface de téléversement
  \item 3- L'officier de police clique sur le champs de l'image faciale
 \item 4- Le system affiche la boite de dialoque
  \item 5- L'officier de police choisit l'image et clique sur "ouvrir"
  \item 6- Le system sauvegarde l'image
  \item 7- L'officier de police refait cette etape 12 fois en totale pour l'iris et pour l'empreinte
  \item 8- Le system sauvegrade tout les images localement
  \item 9- L'officier de police clique sur "submit"
  \item 10- Le system verifie l'etat des image s'ils existent il les sauvegrade dans la base"deduplication" , sinon un numero cpf est généré et les données sont enregistrés dans la base externe
  \item\textbf{Scénario 5: Consulter Déduplications}
  \item 1- Le manager clique sur "déduplications"
  \item 2- Le system charge et affiche lea liste de déduplications
  \item\textbf{Scénario 6: Filtrer Déduplications CPF}
  \item 1- Le manager clique sur les options de filtrage
  \item 2- Le system affiche la liste selon les options choisies
\end{itemize}\\
\hline
\textbf{Enchaînements Alternatifs} &
\begin{itemize}[label=]
\item\textbf{Scénario 1: Consulter CPF}

  \item\textbf{Scénario 4: Téléverser données biométriques}
  \item 1- L'officier de police ne selectionne pas 1 photo facial , 2 iris et 10 empreinte
  \item 2- retourne à l'etape 3 du scénario nominal

\end{itemize} \\



\hline
\textbf{Post-conditions } &
\begin{itemize}[label=]
  \item\textbf{Scénario 4: Téléverser données biométriques}
  \item dés données doivent etre soit sauvegardées dans déduplications, soit sauvegardées dans la base externe de données biométriques
\end{itemize}
\end{longtable}




\subsubsection{Conception}
Dans cette section, nous présentons l'étude conceptuelle des données par la présentation du
diagramme de classes et des diagrammes d'interactions.
\subsubsubsection{Diagramme de classes}
La figure 5.3 décrit le diagramme de classes que nous avons utilisé pour développer le sprint « Gérer CPF ».

\begin{figure}[H]
\centering
\includegraphics[width=\linewidth,height=21cm]{assetschap4/sprint3CPFClass.png}
\caption{ Sprint 3 - Diagramme de classes « Gérer CPF ». }
\end{figure}

\subsubsubsection{Diagrammes d'interaction détaillés}
Dans cette sous-section, nous présentons quelques diagrammes de séquences qui détaillent
l'interaction entre la partie front-end et la partie back-end du sprint « Gérer CPF ».\\




\textbf{– Diagramme d'interaction « Consulter Rendez-vous Plannifiées »}\\
Ce diagramme illustre le processus de consultation des rendez-vous valide de ce jour là afin de téléverser les données biométriques d'un citoyen.
\begin{figure}[H]
\centering
\includegraphics[height = 14 cm , width=15 cm ]{assetschap4/SeqConsulterRendez-vousPlannifiés.drawio (1)_cropped.pdf}
\caption{ Sprint 3 - Diagramme d'interaction « Consulter Rendez-vous Plannifiées ». }
\end{figure}



\textbf{– Diagramme d'interaction « Téléverser Données Biométriques »}\\
La fonction "Téléverser Données Biométriques" permet à un officier de police de téléverser les données biométriques d'un citoyen lors d'un rendez-vous validé. L'officier téléverse les données biométriques(visage, empreintes, iris) puis les soumet au système. Le backend reçoit ces données, les valide, les stocke de manière sécurisée, puis déclenche automatiquement le processus de déduplication et de géneration de CPF .Une fois terminé,le system notifie l'officier du résultat .
\begin{figure}[H]
\centering
\includegraphics[height = 23 cm , width=18 cm ]{assetschap4/SeqTeleverserBiommetriques.drawio (2).png}
\caption{ Sprint 3 - Diagramme d'interaction « Téléverser données biométriques ». }
\end{figure}




\textbf{– Diagramme d'interaction «Consulter CPF »}\\
La fonction "Consulter CPF" permet au citoyen brésilien de consulter les détails de son CPF.
\begin{figure}[H]
\centering
\includegraphics[width=17 cm , height = 9 cm]{assetschap4/seqConsulterMonCPF.drawio (1)_cropped.pdf}
\caption{ Sprint 3 - Diagramme d'interaction « Consulter CPF ». }
\end{figure}





\textbf{– Diagramme d'interaction « Consulter Déduplications »}\\
La fonction "Consulter Déduplication" permet au manager de consulter tous les détails des Déduplications.
\begin{figure}[H]
\centering
\includegraphics[height = 9 cm ,width=17 cm]{assetschap4/SeqConsuDedupp.drawio_cropped.pdf}
\caption{ Sprint 3 - Diagramme d'interaction « Consulter Déduplications ». }
\end{figure}




\textbf{– Diagramme d'interaction « Télécharger CPF »}\\
La fonction "Télécharger CPF" permet de télécharger tous les détails d'un CPF généré.
\begin{figure}[H]
\centering
\includegraphics[height = 22 cm ,width=17 cm]{assetschap4/seqTelechargerCPF.drawio_cropped.pdf}
\caption{ Sprint 3 - Diagramme d'interaction « Télécharger CPF ». }
\end{figure}







\textbf{– Diagramme d'interaction « Filtrer Déduplications »}\\
La fonction "Filtrer Déduplication" permet au manager de filtrer les déduplications afin de faciliter visualiser les détails des records.
\begin{figure}[H]
\centering
\includegraphics[height= 17 cm, width=15 cm]{assetschap4/seqFiltrerDedup.drawio_cropped.pdf}
\caption{ Sprint 3 - Diagramme d'interaction « Filtrer Déduplications ». }
\end{figure}




\subsubsection{Réalisation}
Après avoir analysé le sprint 3 et achevé la phase de conception, cette section présente les
interfaces homme-machine élaborées durant ce sprint.\\










\textbf{– Interfaces du cas d'utilisation « Consulter Rendez-vous Plannifiées »}\\
L'interface consultation des rendez-vous filtrée par la date rélle de ce jour permet à l'officier de police d'avoir la possibilité de téléverser les rendez-vous de ce jour en cliquant sur le bouton "upload"
\begin{figure}[H]
\centering
\includegraphics[width=\linewidth,height=8cm]{assetschap4/cpf realisation/rendez-vousPlannifie.png}
\caption{ Sprint 3 - Interface « Consulter Rendez-vous Plannifiées » }
\label{fig:5.9}
\end{figure}





\textbf{– Interfaces du cas d'utilisation « Téléverser Données Biométriques »}\\
L'interface de collecte biométrique permet à l'officier de police de capturer les différentes données biométriques du citoyen de manière structurée et sécurisée,voir la ~\hyperref[fig:5.9]{figure 5.9}.
L'interface inclut des outils de capture pour la photo du visage, les empreintes digitales "voir la ~\hyperref[fig:5.10]{figure 5.10}" et les scans d'iris "voir la ~\hyperref[fig:5.11]{figure 5.11}", ainsi que des indicateurs de qualité pour s'assurer que les données capturées répondent aux normes requises.
\begin{figure}[H]
\centering
\includegraphics[width=\linewidth,height=8cm]{assetschap4/cpf realisation/face_and_interface.jpeg}
\caption{ Sprint 3 - Interface « téléverser face image » }
\label{fig:5.9}
\end{figure}


\begin{figure}[H]
\centering
\includegraphics[width=\linewidth,height=9cm]{assetschap4/cpf realisation/fingers.jpeg}
\caption{ Sprint 3 - Interface « Téléverser fingerprints » }
\label{fig:5.10}
\end{figure}



\begin{figure}[H]
\centering
\includegraphics[width=\linewidth,height=9cm]{assetschap4/cpf realisation/iris.jpeg}
\caption{ Sprint 3 - Interface « Téléverser iris » }
\label{fig:5.11}
\end{figure}





\textbf{– Interfaces du cas d'utilisation « Télécharger CPF »}\\
Le citoyen peut visualiser son numéro CPF (voir la ~\hyperref[fig:5.12]{figure 5.12}), consulter l'état des générations et télécharger le certificat, comme le montre la ~\hyperref[fig:5.13]{figure 5.13}.
S'il tente une fraude, la ~\hyperref[fig:5.14]{figure 5.14} illustre la différence d'état du citoyen.
\begin{figure}[H]
\centering
\includegraphics[width=\linewidth,height=24cm]{assetschap4/cpf realisation/telecharger cpf_page-0001.jpg}
\caption{ Sprint 3 - Interface « Télécharger CPF » }
\label{fig:5.12}
\end{figure}





\textbf{– Interfaces du cas d'utilisation « Consulter Déduplications »}\\
La fonctionnalité "Consulter Déduplication" permet au manager de consulter les détails des données biométriques dédupliquées .
\begin{figure}[H]
\centering
\includegraphics[width=\linewidth , height = 8 cm]{assetschap4/cpf realisation/consulte_dedup.jpeg}
\caption{ Sprint 3 - Interface « Consulter Déduplications» }
\label{fig:5.12}
\end{figure}


\textbf{– Interfaces du cas d'utilisation « Filtrer Déduplication »}\\
La fonctionnalité "Filtrer Déduplication" permet au manager de Filtrer les déduplications selon pourcentage de similarité , la date , le nom du citoyen ou Id du citoyen.
\begin{figure}[H]
\centering
\includegraphics[width=\linewidth , height = 8 cm]{assetschap4/cpf realisation/filtreNom.png}
\caption{ Sprint 3 - Interface « Filtrer Déduplications» }
\label{fig:5.12}
\end{figure}

\begin{figure}[H]
\centering
\includegraphics[width=\linewidth , height = 8 cm]{assetschap4/cpf realisation/filtrePourcen.png}
\caption{ Sprint 3 - Interface « Filtrer Déduplications» }
\label{fig:5.12}
\end{figure}



\textbf{– Interfaces du cas d'utilisation « Consulter CPF »}\\
La fonctionnalité "Consulter CPF" permet au citoyen brésilien de consulter le statut de son CPF si généré ou frauduleux.

\begin{figure}[H]
\centering
\includegraphics[width=\linewidth,height=12 cm]{assetschap4/cpf realisation/fraud_cpf.jpeg}
\caption{ Sprint 3 - Interface « Consulter CPF en cas "frauduleux"» }
\label{fig:5.13}
\end{figure}




\begin{figure}[H]
\centering
\includegraphics[width=\linewidth,height=14 cm]{assetschap4/cpf realisation/generated_cpf.jpeg}
\caption{ Sprint 3 - Interface « Consulter CPF en cas "généré"» }
\label{fig:5.13}
\end{figure}


































\newpage  % Force une nouvelle page
\subsection{Sprint 4 : « Gérer Transactions et Fraudes »}
Dans cette section, nous décrivons le développement du système de gestion des transactions et fraudes, qui complète notre application d'identité. Ce système permet aux citoyens de consulter leurs transactions basées sur leur CPF, de signaler des fraudes, et aux managers de gérer les cas de fraude détectés.
\subsubsection{Organisation}
Le tableau 5.1 ci-dessous donne un aperçu détaillé sur le Backlog du sprint « Gérer Transactions et Fraudes ».

\begin{longtable}{|>{\centering\arraybackslash}p{0.7cm}|>{\arraybackslash}p{5cm}|>{\centering\arraybackslash}p{1.3cm}|>{\arraybackslash}p{6.5cm}|>{\centering\arraybackslash}p{1cm}|}
\caption{Backlog du sprint 4 : « Gérer Transactions et Fraudes » } \label{tab:backlog} \\
\hline
\rowcolor{gray!30}
\textbf{ID} & \textbf{User Story} & \textbf{ID} & \textbf{Tâche} & \textbf{Durée /j} \\
\hline
\endfirsthead
\hline
\endhead
\hline
\endfoot
\hline
\endlastfoot
% User Story 6.1
\multirow{3}{0.7cm}{6.3} & \multirow{3}{5cm}{En tant que citoyen brésilien, je peux gérer mes transactions basées sur mon CPF, notamment consulter et signaler les frauduleuses.}
& 6.3.1 & Implémenter une interface frontend pour afficher l'historique des transactions liées au CPF du citoyen connecté. & 1.5 \\
\cline{3-5}
& & 6.3.2 & Développer une API backend sécurisée (via JWT) pour récupérer les transactions du citoyen basées sur son CPF. & 1 \\
\cline{3-5}
& & 6.3.3 & Ajouter une fonctionnalité permettant au citoyen de signaler une transaction comme frauduleuse (interface + API). & 1.5 \\
\hline
% User Story 6.2
\multirow{3}{0.7cm}{6.1} & \multirow{3}{5cm}{En tant que manager CPF, je peux consulter et gérer les fraudes liées aux comptes.}
& 6.1.1 & Implémenter une interface frontend pour le manager permettant de consulter les rapports de fraude et transactions signalées. & 1.5 \\
\cline{3-5}
& & 6.1.2 & Développer une API backend pour récupérer et traiter les signalements de fraude (statuts, historique, actions). & 1 \\
\cline{3-5}
& & 6.1.3 & Ajouter des fonctionnalités de gestion des fraudes (blocage/déblocage des comptes et filtrage des fraudes). & 1.5 \\
\hline
% User Story 6.3
\multirow{3}{0.7cm}{6.3} & \multirow{3}{5cm}{En tant que manager CPF ou officier de police, je peux consulter la liste des citoyens avec des filtres par statut (active, suspended, pending) et par état de génération du CPF et par déduplications.}
& 6.3.1 & Implémenter une interface frontend avec filtres avancés pour consulter la liste des citoyens (statut, état CPF, doublons détectés). & 1.5 \\
\cline{3-5}
& & 6.3.2 & Développer une API backend pour la gestion des citoyens avec filtrage. & 1 \\
\cline{3-5}
& & 6.3.3 & Implémenter le système de détection et gestion des doublons/déduplications avec interface de résolution. & 2 \\
\hline
\end{longtable}

\subsubsection{Analyse}
Durant cette phase d'analyse, nous approfondissons les diverses fonctionnalités en les accompagnant de leurs cas d'utilisation respectifs.

\textbf{– Raffinement de cas d'utilisation « Gérer Transactions et Fraudes »}\\
La Figure 5.1 montre le raffinement de cas d'utilisation « Gérer Transactions et Fraudes» mettant en lumière les interactions entre les citoyens, les officiers et les managers dans le processus de gestion des CPF.

\begin{figure}[H]
\centering
\includegraphics[scale=0.65]{assetschap4/GérerTransEtFraudeUC.drawio_cropped.pdf}
\caption{ Sprint 3 - Diagramme de cas d'utilisation « Gérer Transactions et Fraudes ». }
\end{figure}

Le Tableau 5.2 représente une description textuelle du cas d'utilisation « Gérer Transactions et Fraudes ». Il
détaille le scénario nominal ainsi que les enchaînements alternatifs.

\begin{longtable}{|>{\arraybackslash}p{4.2cm}|>{\arraybackslash}p{12.5cm}|}
\caption{\centering Description textuelle du cas d'utilisation « Gérer Transactions et Fraudes »}
\label{tab:backlog} \\
\hline
\rowcolor{gray!30}
\textbf{Cas d'utilisation} & Gérer Transactions et Fraudes \\
\hline
\endfirsthead

\hline
\endhead

\hline
\endfoot

\hline \hline
\endlastfoot
 \textbf{Acteur}  & Citoyen, Officier de Police, Manager CPF\\
\hline
\textbf{Résumé} &
\begin{itemize}[label=]
  \item\textbf{Scénario 1: Gérer Transactions Basées CPF}
  \item Le citoyen brésilien gère ses transactions basées sur son CPF.
  \item\textbf{Scénario 2: Consulter Transactions}
  \item Le citoyen brésilien peut consulter ses transactions.
  \item\textbf{Scénario 3: Signaler Transactions}
  \item Le citoyen brésilien peut signaler une transaction suspecte.
  \item\textbf{Scénario 4: Consulter Liste Citoyens}
  \item Le manager et l'officier de police peuvent consulter la liste des citoyens.
  \item\textbf{Scénario 5: Filtrer Liste Citoyens}
  \item Le manager et l'officier de police peuvent filtrer la liste des citoyens.

\end{itemize}\\

\hline


\textbf{}&
\begin{itemize}[label=]
\item\textbf{Scénario 6: Gérer Fraudes}
  \item Le manager peut gérer les fraudes.
  \item\textbf{Scénario 7: Consulter Fraudes}
  \item Le manager peut consulter les fraudes.
  \item\textbf{Scénario 8: Filtrer Fraudes}
  \item Le manager peut filtrer la liste des fraudes.
  \item\textbf{Scénario 9: Bloquer compte bancaire}
  \item Le manager peut bloquer un compte bancaire déclaré.
  \item\textbf{Scénario 10: Débloquer compte bancaire}
  \item Le manager peut débloquer un compte bancaire.

\end{itemize}\\
\hline
\textbf{Pré-conditions} &
\begin{itemize}[label=]
 \item\textbf{Scénario 1: Gérer Transactions Basées CPF}
  \item Le citoyen brésilien est authentifié.
    \item Le citoyen brésilien clique sur "Transactions".
  \item\textbf{Scénario 2: Consulter Transactions}
  \item Le scénario 1 « Gérer Transactions Basées CPF » est bien exécuté.
  \item\textbf{Scénario 3: Signaler Transactions}
   \item Le scénario 2 « Consulter Transactions » est bien exécuté.
     \item Le système affiche l'interface.
  \item\textbf{Scénario 4: Consulter Liste Citoyens}
  \item Le manager et l'officier de police sont authentifiés.
    \item Le manager et l'officier de police cliquent sur "citizens".

\end{itemize} \\
\hline
\textbf{}&
\begin{itemize}[label=]
  \item\textbf{Scénario 5: Filtrer Liste Citoyens}
  \item Le scénario 4 « Consulter Liste Citoyens » est bien exécuté.
  \item Le système affiche l'interface.
  \item\textbf{Scénario 6: Gérer Fraudes}
  \item Le manager est authentifié.
  \item Le manager clique sur "Frauds".
   \item\textbf{Scénario 7: Consulter Fraudes}
  \item Le scénario 6 « Gérer Fraudes » est bien exécuté.
   \item\textbf{Scénario 8: Filtrer Fraudes}
   \item Le scénario 7 « Consulter Fraudes » est bien exécuté.
    \item Le système affiche l'interface.
   \item\textbf{Scénario 9: Bloquer compte bancaire}
   \item Le scénario 7 « Consulter Fraudes » est bien exécuté.
   \item Le système affiche l'interface.
   \item\textbf{Scénario 10: Débloquer compte bancaire}
   \item Le scénario 7 « Consulter Fraudes » est bien exécuté.
    \item Le système affiche l'interface.
\end{itemize}\\
\hline
\textbf{Description de scénario nominal}&
\begin{itemize}[label=]
  \item\textbf{Scénario 1: Gérer Transactions Basées CPF}
  \item Le système fait appel au cas d'utilisation "Consulter Transactions".
  \item\textbf{Scénario 2: Consulter Transactions}
  \item Le citoyen brésilien peut consulter ses transactions.
  \item\textbf{Scénario 3: Signaler Transactions}
  \item Le citoyen brésilien peut signaler une transaction suspecte.
\end{itemize}
\hline





\textbf{} &
\begin{itemize}[label=]
  \item\textbf{Scénario 4: Consulter Liste Citoyens}
  \item Le manager et l'officier de police accèdent à l'interface "citizens".
  \item{2-} Le système affiche la liste des citoyens.
  \item 3-L’acteur parcourt la liste des citoyens pour consulter les détails.
  \item 4- L’acteur sélectionne un citoyen spécifique pour obtenir plus d’informations si nécessaire
  \item\textbf{Scénario 5: Filtrer Liste Citoyens}
  \item L'acteur clique sur l'option de filtrage.
  \item Le système affiche la liste d'options de filtrage.
  \item L'acteur choisit le critère désiré.
  \item\textbf{Scénario 6: Gérer Fraudes}
  \item Le système fait appel au cas d'utilisation "Consulter Fraudes".
   \item\textbf{Scénario 7: Consulter Fraudes}
  \item Le manager accède à l'interface "Frauds".
  \item{2-} Le système affiche la liste des fraudes.
  \item 3-L’acteur parcourt la liste des fraudes pour consulter les détails.
   \item\textbf{Scénario 8: Filtrer Fraudes}
   \item L'acteur choisit les critères de filtrage.
    \item Le système affiche la liste filtrée.

   \item\textbf{Scénario 9: Bloquer comptes bancaires}
  \item Le manager clique sur le bouton "Block Account".
   \item\textbf{Scénario 10: Débloquer comptes bancaires}
  \item Le manager clique sur le bouton "Unblock Account".
\end{itemize} \\

\hline
\textbf{Enchaînements Alternatifs} &
\begin{itemize}[label=]
  \item\textbf{Scénario 3: Signaler Transactions}
  \item Le citoyen Brésilien ne saisit pas la cause ou ne choisit pas le type de fraude ou annule l'envoie de la demande .
   \end{itemize} \\
\hline
\textbf{Post-conditions } &
\begin{itemize}[label=]
   \item\textbf{Scénario 9: Bloquer compte banquaire}
  \item Le compte est met à jour en bloqué dans la base de données.
   \item\textbf{Scénario 10: Débloquer compte banquaire}
   \item Le compte est met à jour en active dans la base de données.
\end{itemize} \\
\end{longtable}

\subsubsection{Conception}
Dans cette section, nous présentons l'étude conceptuelle des données par la présentation du
diagramme de classes et des diagrammes d'interactions.
\subsubsection{Diagramme de classes}
La figure 5.3 décrit le diagramme de classes que nous avons utilisé pour développer le sprint « Gérer Transactions et fraudes ».

\begin{figure}[H]
\centering
\includegraphics[width=18 cm, height= 25cm]{assetschap4/transFrauds/sprint4fraudTransClass.drawio.png}
\caption{ Sprint 3 - Diagramme de classes « Gérer Transactions et fraudes ». }
\end{figure}

\textbf{Diagrammes d'interaction détaillés}
Dans cette sous-section, nous présentons quelques diagrammes de séquences qui détaillent
l'interaction entre la partie front-end et la partie back-end du sprint « Gérer Transactions et fraudes ».\\




\textbf{– Diagramme d'interaction «Gérer Fraudes »}\\
La fonction "Gérer Fraudes" permet au manager de gérer les fraudes .
\begin{figure}[H]
\centering
\includegraphics[width = 17 cm , height = 7 cm]{assetschap4/transFrauds/seqGérerFrauds.drawio (1)_cropped.pdf}
\caption{ Sprint 3 - Diagramme d'interaction « Gérer Fraudes ». }
\end{figure}




\textbf{– Diagramme d'interaction «Consulter Fraudes »}\\
La fonction "Consulter Fraudes" permet au manager de consulter les détails des fraudes .
\begin{figure}[H]
\centering
\includegraphics[width = 17 cm , height = 8 cm]{assetschap4/transFrauds/SeqConsulterFrauds.drawio_cropped.pdf}
\caption{ Sprint 3 - Diagramme d'interaction « Consulter Fraudes ». }
\end{figure}




\textbf{– Diagramme d'interaction «Bloquer Compte Banquaire »}\\
La fonction "Bloquer Compte Banquaire" permet au manager de bloquer les comptes banquaires selon une demande recu de la part d'un citoyen.
\begin{figure}[H]
\centering
\includegraphics[width = 17 cm , height = 8 cm]{assetschap4/transFrauds/SeqBlockAccount(Frauds).drawio_cropped.pdf}
\caption{ Sprint 3 - Diagramme d'interaction « Bloquer Compte Banquaire ». }
\end{figure}




\textbf{– Diagramme d'interaction «Débloquer Compte Banquaire»}\\
La fonction "Débloquer Compte Banquaire" permet au manager de débloquer un compte banquaire.
\begin{figure}[H]
\centering
\includegraphics[width = 17 cm , height = 9 cm]{assetschap4/transFrauds/seqUnblock(fraud).drawio (1)_cropped.pdf}
\caption{ Sprint 3 - Diagramme d'interaction « Débloquer Compte Banquaire ». }
\end{figure}




\textbf{– Diagramme d'interaction «Gérer Transactions Basées CPF »}\\
La fonction "Gérer Transactions Basées CPF" permet au cityen brésilien de gére son numéro CPF tout en ayant l'accées à ses transactions.
\begin{figure}[H]
\centering
\includegraphics[width = 17 cm , height = 8 cm]{assetschap4/transFrauds/SeqTransactionBaseeCPF.drawio_cropped.pdf}
\caption{ Sprint 3 - Diagramme d'interaction « Gérer Transactions Basées CPF ». }
\end{figure}






\textbf{– Diagramme d'interaction «Consulter Transactions »}\\
La fonction "Consulter Transactions" permet au cityen brésilien de consulter ses transactions liées au comptes banquaires de son numéro CPF.
\begin{figure}[H]
\centering
\includegraphics[width = 17 cm , height = 9 cm]{assetschap4/transFrauds/SeqConsulterMesTransactions.drawio_cropped.pdf}
\caption{ Sprint 3 - Diagramme d'interaction « Gérer Transactions Basées CPF ». }
\end{figure}




\textbf{– Diagramme d'interaction «Signaler Transactions »}\\
La fonction "Signaler Transactions" permet au citoyen brésilien d'envoyer une demande de signale en cas de transaction frauduleuse .
\begin{figure}[H]
\centering
\includegraphics[width = 17 cm , height = 21 cm]{assetschap4/transFrauds/SeqSignalerTransaction.drawio (1)_cropped.pdf}
\caption{ Sprint 3 - Diagramme d'interaction « Signaler Transactions ». }
\end{figure}





\textbf{– Diagramme d'interaction «Consulter Citoyens »}\\
La fonction "Consulter Citoyens" permet au manager et à l'officier de police de consulter tout les détails des citoyens .
\begin{figure}[H]
\centering
\includegraphics[width = 17 cm , height = 17 cm]{assetschap4/transFrauds/SeqConsulterCitizens.drawio_cropped.pdf}
\caption{ Sprint 3 - Diagramme d'interaction « Consulter Citoyens ». }
\end{figure}





\textbf{– Diagramme d'interaction «Filtrer Citoyens »}\\
La fonction "Filtrer" permet au manager et à l'officier de police de filtrer la liste  des citoyens selon plusieurs critéres.
\begin{figure}[H]
\centering
\includegraphics[width = 17 cm , height = 14 cm]{assetschap4/transFrauds/seqFiltrerCitoyens.drawio (1)_cropped.pdf}
\caption{ Sprint 3 - Diagramme d'interaction « Filtrer Citoyens ». }
\end{figure}




\subsubsection{Réalisation}
Après avoir analysé le sprint 3 et achevé la phase de conception, cette section présente les
interfaces homme-machine élaborées durant ce sprint.\\
%A3ml hadhi
comme le montre la~\hyperref[fig:5.27]{ figure 5.27}


\textbf{– Interfaces du cas d'utilisation « Consulter frauds »}\\
L'interface de consultation des fraudes permet au manager de gérer les fraudes en ayant la possibilité pour bloquer et débloquer les comptes banquaires et pour visualiser les détails de chaque demande de fraudes.
\begin{figure}[H]
    \centering
    \includegraphics[width= 17 cm , height = 24 cm]{assetschap4/transFrauds/realisation/consulte fraud.png}
    \caption{ Sprint 4 - Interface « consulter frauds »}
    \label{fig:5.27}
\end{figure}








\textbf{– Message de succées lors de blocage d'un compte banquaire}\\

\begin{figure}[H]
    \centering
    \includegraphics[]{assetschap4/transFrauds/realisation/blocked.png}
    \caption{ Sprint 4 - Interface « consulter frauds »}
    \label{fig:5.27}
\end{figure}

\textbf{– Message de succées lors de déblocage d'un compte banquaire}\\

\begin{figure}[H]
    \centering
    \includegraphics[]{assetschap4/transFrauds/realisation/unblocked.png}
    \caption{ Sprint 4 - Interface « consulter frauds »}
    \label{fig:5.27}
\end{figure}













\textbf{– Interfaces du cas d'utilisation « Filtrer frauds »}\\
L'interface de filtrage des frauds selon plusieurs critére tel que "Risk Level" et "Status" .
\begin{figure}[H]
    \centering

    \includegraphics[ width = 17 cm , height = 12 cm ]{assetschap4/transFrauds/realisation/filtre.png}
    \caption{ Sprint 4 - Interface « Filtrer frauds »}
    \label{fig:5.27}
\end{figure}


\textbf{– Interfaces du cas d'utilisation « Consulter Transactions »}\\
L'interface de consultation des transactions permet au citoyen brésilien de visualiser ses transactions associées à ses comptes banquaires. Elle offre également des fonctionnalités pour demande un blocage en cas de transactions frauduleuse tout en cliquant sur le bouton "Report Fraud".
\begin{figure}[H]
    \centering
    \includegraphics[width=1\linewidth  ]{assetschap4/transFrauds/realisation/consulte transaction.png}
    \caption{Sprint 4 - Interface « consulter transactions »}
    \label{fig:5.28}
\end{figure}




\textbf{– Interfaces du cas d'utilisation «signaler Transactions »}\\
L'interface du signale de transaction permet au citoyen brésilien de reporter une transaction frauduleuse tout en choisissant le type de fraud et saisissant un commentaire , aprés avoir remplit le formilaire , le citoyen brésilien clique sur le bouton  "Submit Report".
\begin{figure}[H]
    \centering
    \includegraphics[ ]{assetschap4/transFrauds/realisation/repport transaction.jpeg}
    \caption{Sprint 4 - Interface « signaler transactions »}
    \label{fig:5.29}
\end{figure}




\textbf{– Interfaces du cas d'utilisation « Consulter liste citoyens »}\\
L'interface de consultation des citoyens permet au manager CPF et à l'officier de police de visualiser et de filtrer la liste complète des citoyens. Elle offre des fonctionnalités de pagination et de filtrage par statut (actif, suspendu, bloqué) et par état de génération du CPF (en attente, généré, frauduleux). L'interface permet également de sélectionner un citoyen pour afficher ses informations détaillées, y compris son historique de statut et les éventuels cas de déduplication ou de fraude associés,comme le montre la ~\hyperref[fig:5.15]{figure 5.15}.

\begin{figure}[H]
\centering
\includegraphics[width=\linewidth]{assetschap4/cpf realisation/consulte citizens.jpeg}
\caption{ Sprint 3 - Interface « Consulter liste citoyens » }
\label{fig:5.14}
\end{figure}

\addcontentsline{toc}{section}{Conclusion}
\section*{Conclusion}
\label{sec_Conclusion:ch4}
Lors de cette release, nous avons accompli l'analyse, la conception et la réalisation du sprint « Gérer CPF ». À ce stade, nous avons mis en place un système complet de gestion des identités brésiliennes, permettant la collecte sécurisée des données biométriques, la vérification automatique de l'unicité des identités, et la génération des numéros CPF. Le processus automatisé de déduplication et de génération de CPF constitue le cœur de notre application, offrant une solution fiable et sécurisée pour l'identification des citoyens brésiliens.

Les interfaces développées permettent aux différents acteurs (citoyens, officiers, managers) d'interagir efficacement avec le système selon leurs rôles et responsabilités. Les citoyens peuvent consulter et télécharger leurs CPF, les officiers peuvent collecter les données biométriques lors des rendez-vous, et les managers et officiers peuvent consulter la liste des citoyens avec différents filtres pour suivre l'état de génération des CPF et les statuts des comptes.

Cette release représente une étape cruciale dans le développement de notre système d'identité, fournissant les fonctionnalités essentielles pour la gestion des CPF et posant les bases pour les futures améliorations et extensions du système.
