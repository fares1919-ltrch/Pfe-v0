\chapter*{Conclusion \& Perspectives}
\addcontentsline{toc}{chapter}{Conclusion générale}
La réalisation du projet « Identity Secure » marque une avancée majeure dans la transformation numérique et sécurisée de la gestion des numéros d'identification fiscale brésilien (CPF). Grâce à une méthodologie agile basée sur Scrum et à l’intégration de technologies de pointe, nous avons développé une solution robuste, évolutive et hautement sécurisée, répondant efficacement aux besoins des utilisateurs, des officiers de police et des managers. Ce projet constitue une étape clé vers une administration plus intelligente, plus rapide et plus transparente, ouvrant la voie à une génération de services publics numériques.

L’innovation technologique a été placée au cœur du projet.En intégrant l’intelligence artificielle, notamment pour la détection des fraudes et la reconnaissance biométrique intelligente. La sécurité est assurée par l’utilisation du chiffrement AES-256 pour les données en transit et au repos, ainsi que du hachage SHA-256 pour la protection des identifiants. De plus, une authentification multifactorielle (MFA) combinant biométrie et jetons dynamiques a été mise en œuvre afin de renforcer l’accès sécurisé. Enfin, l’architecture décentralisée fondée sur la blockchain garantit la traçabilité des transactions, tout en assurant la conformité avec la LGPD.

En regardant vers l’avenir, notre objectif est de consolider ces acquis en améliorant continuellement les performances, la scalabilité et l’ergonomie de la plateforme. Nous envisageons d’élargir le périmètre fonctionnel à d’autres types d’identification numérique et d’adapter la solution à de nouveaux cas d’usage à l’échelle nationale et internationale. Nous continuerons également à affiner l’expérience utilisateur, avec une interface toujours plus intuitive, multilingue et accessible, ainsi que des outils de reporting interactifs pour permettre un suivi et une prise de décision optimisés.

Nous restons pleinement mobilisés pour faire d’Identity Secure une solution de référence, alliant technologie de confiance, innovation continue et respect des normes, afin de garantir une gestion d’identité numérique plus efficace et résolument tournée vers l’avenir.

\markboth{Conclusion générale}{}
