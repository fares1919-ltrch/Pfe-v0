\chapter*{Introduction générale}
\addcontentsline{toc}{chapter}{Introduction générale} % to include the introduction to the table of content
\noindent{ \textbf{D}}ans un contexte où la transformation numérique s’accélère à l’échelle mondiale, la sécurisation de l’identité numérique devient un enjeu stratégique majeur pour les États et les citoyens. Cette transition digitale, bien qu’indispensable, expose les systèmes d’identification à des risques accrus tels que la falsification d'identité, les cyberattaques, les fuites de données personnelles et l’exploitation frauduleuse des informations fiscales.

Face à ces menaces croissantes, il devient impératif de revoir les mécanismes traditionnels de gestion d’identité, souvent fondés sur des systèmes centralisés et des vérifications manuelles dépassées. Dans ce contexte critique, le contrôle, la sécurité et l’authentification rigoureuse des identifiants sont au cœur des priorités pour garantir la confiance numérique et assurer la sécurité des données biométriques.

C’est dans cette dynamique que s’inscrit le projet "Identity Secure", une solution innovante développée par l’équipe T4isb. Cette plateforme vise à moderniser et à sécuriser la gestion des identifiants fiscaux (CPF)\cite{b42} au Brésil grâce à l’intégration de technologies de pointe telles que la biométrie, la cryptographie\cite{b46}, la déduplication intelligente et la blockchain\cite{b47}. En automatisant la vérification, en assurant la traçabilité des identités et en facilitant la détection des fraudes en temps réel, Identity Secure entend transformer les processus actuels en un écosystème numérique fiable, décentralisé et conforme aux normes de protection des données (LGPD)\cite{b48}.

Ce projet répond à un besoin urgent d’adapter les infrastructures d’identification aux exigences de sécurité modernes, tout en renforçant la transparence, la sécurité et l’efficacité dans la délivrance et le contrôle des identités numériques au Brésil.\\

Le présent mémoire synthétise l'ensemble du travail accompli dans cette perspective avec méthodologie Scrum. En dehors de l'introduction et de la conclusion, il est structuré en cinq chapitres :
\\
% \vspace{0.9 cm}
\textbf{- Chapitre 1 « Analyse et Spécification des Besoins »:}
Dans ce chapitre, nous abordons le contexte et la problématique du projet, en passant en revue les solutions existantes et en définissant les besoins fonctionnels et non fonctionnels de notre plateforme.\\
\textbf{- Chapitre 2 « Phase de Préparation »:}
Ce chapitre se concentre sur la méthodologie Scrum adoptée pour piloter le projet, en détaillant l'organisation de l'équipe, la création du Product Backlog et la planification des sprints.\\
\textbf{- Chapitre 3 « Release 1 :« Gérer Utilisateurs » »:}
Nous présentons la première version de notre application, axée sur la gestion des comptes utilisateurs et des rôles, en détaillant les étapes d'analyse, de conception et de réalisation des sprints « Gérer Rôles » et « Gérer Utilisateurs ».\\
\textbf{- Chapitre 4 « Release 2 :« Gérer Rendez-vous»:}
Dans ce chapitre, nous décrivons les fonctionnalités développées lors du sprint  « Gérer Rendez-vous », visant à gérer les demandes des Rendez-vous avec des diagrammes et des implémentations détaillés.\\
\textbf{- Chapitre 5 « Release 3 :« Gérer CPF \& transactions et fraudes » »:}
Dans ce chapitre, nous décrivons les fonctionnalités développées lors des sprints  « Gérer CPF » et  « Gérer Transactions et Fraudes », visant à gérer les numéro cpf, les transactions, et les fraudes, avec des diagrammes et des implémentations détaillés.\\
\textbf{- Chapitre 6 « Release 4 :« Procurer Aide \& Gérer notifications » »:}
Le dernier chapitre se penche sur les fonctionnalités additionnelles visant à renforcer l'interaction utilisateur et à fournir une aide efficace, détaillant les aspects organisationnels et techniques du sprint  « Procurer Aide » et « Gérer notifications ».\\
Ce mémoire de projet de fin d’études se conclut par une brève synthèse et des perspectives futures.
\markboth{Introduction générale}{} %To redefine the section page head
